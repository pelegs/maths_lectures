\documentclass[a4paper]{report}
\setlength{\parindent}{0pt}

\usepackage[margin=1.5cm]{geometry}
\usepackage[bottom]{footmisc}
\usepackage{fancyhdr}
\usepackage{hyperref}
\usepackage{amsmath}
\usepackage{amssymb}
\usepackage{amsthm}
\usepackage{cancel}
\usepackage{xstring}
\usepackage{mathtools}
\usepackage{tabu}
\usepackage{enumitem}
\usepackage{float}
\usepackage[table]{xcolor}
\usepackage[most]{tcolorbox}
\usepackage{forloop}

\usepackage{pythontex}

\usepackage{pgf}
\usepackage{pgfplots}
\usepgfplotslibrary{fillbetween}
\pgfplotsset{every axis/.append style={
                    font=\tiny,
                    axis x line=middle,
                    axis y line=middle,
                    axis line style={<->, thick, black},
                    xlabel={$x$},
                    ylabel={$y$},
                    every axis x label/.style={at={(current axis.right of origin)}, xshift=10pt, anchor=east},
                    every axis y label/.style={at={(current axis.above origin)}, yshift=10pt, anchor=north},
                    every extra x tick/.style={xticklabel style={above}},
                    every extra y tick/.style={yticklabel style={right}}
}}

\usepackage{tikz}
\usetikzlibrary{calc, positioning, decorations.pathreplacing, shapes, backgrounds, arrows, automata, fit}
\tikzset{My Node Style/.style={midway, right, xshift=3.0ex, align=left, font=\small, draw=none, thin, text=black}}
\tikzset{%
  highlight/.style={rectangle, rounded corners, draw, fill opacity=0.5, inner sep=0pt}
}
\newcommand{\tikzmark}[1]{
  \tikz[overlay, remember picture, baseline=(#1.base)] \node (#1) {};
}

\newcommand{\Highlight}[2][submatrix]{%
    \tikz[overlay,remember picture]{
    \node[highlight, fill=#2, fit=(left.north west) (right.south east)] (#1) {};}
}
\newcommand\VerticalBrace[4][]{%
    % #1 = draw options
    % #2 = top mark
    % #3 = bottom mark
    % #4 = label
	\begin{tikzpicture}[overlay,remember picture]
	  \draw[decorate,decoration={brace, amplitude=1.5ex}, #1] 
	    ([yshift=1ex]#2.north east)  -- ([yshift=-1ex]#3.south east)
	        node[My Node Style] {#4};
	\end{tikzpicture}
}

\definecolor{light-gray}{gray}{0.95}

\newcommand*{\rom}[1]{\expandafter\@slowromancap\romannumeral #1@}

\newcommand{\dd}[1]{
  \frac{d}{d#1}
}

\newcommand\rotMat[1]{%
    \begin{bmatrix}
      \cos\left( #1 \right) & -\sin\left( #1 \right) \\
      \sin\left( #1 \right) &  \cos\left( #1 \right) \\
    \end{bmatrix}
}

\newcommand\digits{} % just for safety
\def\digits[#1](#2:#3:#4:#5)% [draw options](size:base:number)
{
    \pgfmathsetmacro{\n}{#5}
    \draw[#1] (0,0) rectangle (#2*\n,#2);
    \foreach \b in {1,...,\n}{
        \draw[#1,-] (#2*\b,0) to (#2*\b,#2);
        \pgfmathtruncatemacro{\A}{\n-\b}
        \node at (#2*\b-#2/2,-0.25) {$#3^{\A}$};
        \node at (#2*\b-#2/2,0.35) {$\StrChar{#4}{\b}$};
    }
}

\pgfdeclarelayer{background}
\pgfsetlayers{background,main}

\renewcommand\CancelColor{\color{red!70}}

\newsavebox{\pycodebox}

\newcommand\issolution[1]{%
  \ifdefined\withsol
    #1
  \else
  \fi
}

\newcommand{\drawvec}[5]{%
  \coordinate (vec) at (#1,#2);
  \draw[-, dashed, gray] (#1,#2) to (0,#2);
  \draw[-, dashed, gray] (#1,#2) to (#1,0);
  \draw[->, #3, very thick] (0,0) to (vec);
  \node[#3, #4 = 0.1cm of vec] {\Large #5};
}

\newcommand{\highlight}[2][red]{\mathchoice%
  {\colorbox{#1}{$\displaystyle#2$}}%
  {\colorbox{#1}{$\textstyle#2$}}%
  {\colorbox{#1}{$\scriptstyle#2$}}%
{\colorbox{#1}{$\scriptscriptstyle#2$}}
}%

\newcommand{\solutionhead}{

}

\newenvironment{answer}
  {
    \begin{tcolorbox}[breakable, enhanced]
  }
  {
    \end{tcolorbox}
  }

\pagestyle{fancy}
\lhead{Computer Science and Mathematics (B.MES.108), SoSe19}

\title{Computer Science and Mathematics (B.MES.108)\\Tutorial, Summer Semester 2019\\}
\author{Peleg Bar Sapir}

\begin{document}
\maketitle
\setcounter{tocdepth}{2}
\chapter{Exercises}
\def\withsol{0}
\newcounter{ex}
\newcounter{probnum}
\forloop{ex}{1}{\value{ex} < 10}{
  \setcounter{probnum}{0}
  \def\exnum{\arabic{ex}}
  \rhead{Exercise \exnum}
  \include{ex\arabic{ex}}
  \newpage
}
\setcounter{probnum}{0}
\section{Limits of Real Functions}
Calculate the following limits:
\begin{enumerate}
	\item $\limit{x}{\pm\infty}x^{5}-3,\ \limit{x}{\pm\infty}4x^{3}-2x^{7}+103x^{5}$.
		\if\withsol1{
				\begin{answer}
					For a big enough $x$, $x^{5}\gg-3$ (the symbol $\gg$ means \textit{much bigger than}), and thus $\limit{x}{\infty}x^{5}-3=\infty$. The same argument works for a large negative $x$, although in this case it doesn't matter since we are subtracting $3$. Thus $\limit{x}{-\infty}x^{5}-3=-\infty$.\\

					Generally, when we are faced with a limit at $\pm\infty$ of a polynomial expression, the largest power of $x$ is the only thing that matters. Thus, for example, in $\limit{x}{\pm\infty}4x^{3}-2x^{7}+103x^{5}$ the only component of the polynomial that matters is $-2x^{7}$. Since $\limit{x}{\pm\infty}-2x^{7}=\mp\infty$ (i.e. when $x\rightarrow\infty,\ -2x^{7}\rightarrow-\infty$, and when $x\rightarrow-\infty,\ -2x^{7}\rightarrow\infty$), we get $\limit{x}{\pm\infty}4x^{3}-2x^{7}+103x^{5}=\mp\infty$.
				\end{answer}
			}\fi
		\item $\limit{x}{\pm\infty}\frac{1}{x},\ \limit{x}{ 0}\frac{1}{x},\ \limit{x}{\pm\infty}\frac{1}{x^{2}},\ \limit{x}{ 0}\frac{1}{x^{2}}$
			\if\withsol1{
					\begin{answer}
						The bigger the value of $x$, the smaller the expression $\frac{1}{x}$ gets. The same is of course true for large negative values of $x$, the only difference being that the values approach $0$ from the negative numbers. Thus, $\limit{x}{\pm\infty}\frac{1}{x}=0$. The opposite occurs for small values of $x$: as $x$ approaches $0$ the expression $\frac{1}{x}$ becomes larger and larger.

						However, the sign of $x$ plays a role, as $1$ over a positive number is positive, while $1$ over a negative number is negative. Thus when approaching $0$ from the positive numbers the limit tends towards $\infty$, while when approaching $0$ from the negative numbers the limits will go to $-\infty$. In mathematical notation:
						\begin{equation*}
							\limit{x}{0^{+}}\frac{1}{x}=\infty,\ \limit{x}{0^{-}}\frac{1}{x}=-\infty.
						\end{equation*}

						Therefore, the limit at $0$ just does not exist. For more insight, look at a graph of $f\left( x \right)=\frac{1}{x}$ (Figure \ref{fig:one_over_x}).
						\begin{figure}[H]
							\centering
							\begin{tikzpicture}
								\begin{axis}[
										Axis Style,
										xmin=-5, xmax=5,
										ymin=-5, ymax=5,
										domain=-5:5,
									]
									\addplot[function, col1!75] {1/x};
								\end{axis}
							\end{tikzpicture} 
							\caption{$f\left( x \right)=\frac{1}{x}$ for $x\in\left[ -5,5 \right]$.}
							\label{fig:one_over_x}
						\end{figure}

						The same analysis is true for $f\left( x \right)=\frac{1}{x^{2}}$ except that since $x^{2}$ is always non-negative, so is $\frac{1}{x^{2}}$, and thus the graph looks as in Figure \ref{fig:one_over_x_square} (notice also how $\frac{1}{x^{2}}$ diverges/decays much faster than $\frac{1}{x}$).

						Of course, since
						\begin{equation*}
							\limit{x}{0^{-}}\frac{1}{x^{2}}=\limit{x}{0^{+}}\frac{1}{x^{2}},
						\end{equation*}
						the limit at $0$ is defined and simply equals $0$.
						\begin{figure}[H]
							\centering
							\begin{tikzpicture}
								\begin{axis}[
										Axis Style,
										xmin=-5, xmax=5,
										ymin=-5, ymax=5,
										domain=-5:5,
									]
									\addplot[function, col2!75] {1/x^2};
								\end{axis}
							\end{tikzpicture} 
							\caption{$f\left( x \right)=\frac{1}{x^{2}}$ for $x\in\left[ -5,5 \right]$.}
							\label{fig:one_over_x_square}
						\end{figure}
					\end{answer}
				}\fi

			\item $\limit{x}{\infty}\frac{x^{4}-3x^{2}+10x}{-2x^{2}-5},\ \limit{x}{ -1}\frac{2x^{2}+x-1}{x+1}$
				\if\withsol1{
						\begin{answer}
							The highest power of $x$ in both the numerator and the denominator is $x^{4}$, so it controls the behavior of the function as $x\rightarrow\infty$. Since the coefficient of $x^{4}$ is positive (it is simply $1$), we get
							\begin{equation*}
								\limit{x}{\infty}\frac{x^{4}-3x^{2}+10x}{-2x^{2}-5}=\infty.
							\end{equation*}

							The other expression, $\frac{2x^{2}+x-1}{x+1}$, is somewhat tricky. If we pay close attention we can see that $2x^{2}+x-1=\left( 2x-1 \right)\left( x+1 \right)$, and thus
							\begin{align*}
								\frac{2x^{2}+x-1}{x+1} &= \frac{\left( 2x-1 \right)\cancel{\left( x+1 \right)}}{\cancel{x+1}}\\
								&= 2x-1.
							\end{align*}
							so we expect the two expressions to behave in the same way (i.e. be a simple line). However, $\frac{2x^{2}+x-1}{x+1}$ is still undefined at $x=-1$, and thus the function $f\left( x \right)=\frac{2x^{2}+x-1}{x+1}$ has a 'hole' at $x=-1$.

							Except for this point, however, it is well-behaved and looks exactly like $2x-1$, and thus 
							\begin{equation*}
								\limit{x}{ -1}\frac{2x^{2}+x-1}{x+1}=\limit{x}{ -1}2x-1=-3,
							\end{equation*}
							even though $\frac{2x^{2}+x-1}{x+1}$ is undefined at $x=-1$ (limits do not care about values at specific points! They only care about the behaviour leading to a point).
						\end{answer}
					}\fi

				\item $\limit{x}{\pm\infty}\frac{P_{n}\left( x \right)}{P_{m}\left( x \right)}$, where $P_{k}\left( x \right)$ is a real polynomial of order $k$, $n$ is even, $m$ is odd and $n>m$.\\
					\small{\underline{Note}: A real polynomial $P_{k}\left( x \right)$ is defined as $P_{k}\left( x \right)=\sum\limits_{i=0}^{k}a_{i}x^{i}$, with $a_{i}\in\mathbb{R}$ and $a_{k}\neq0$.}
					\if\withsol1{
							\begin{answer}
								Let's generalize what we saw in the previous paragraph: since for any polynomial the limit at $\pm\infty$ is depended only on the term with the highest power of $x$, we can write
								\begin{align*}
									\lim_{x\rightarrow\pm\infty}\frac{P_n\left( x \right)}{P_m\left( x \right)} &= \lim_{x\rightarrow\pm\infty}\frac{a_{n}x^{n}}{b_{n}x^{m}}\\
									&= \lim_{x\rightarrow\pm\infty}\frac{a_{n}}{b_{n}}x^{n-m}.
								\end{align*}
								where $a_{n}$ and $b_{n}$ are the coefficients of the terms $x^{n}$ and $x^{m}$ for $P_n\left( x \right)$ and $P_m\left( x \right)$, respectively.\\
								We can see that there are three possibilities:
								\begin{itemize}
									\item $n>m$: the limit would be $\pm\infty$, depending on the sign of $\frac{a_{n}}{b_{n}}$.
									\item $n=m$: in this case $x^{n-m}=x^{0}=1$, and thus the limit would be $\frac{a_{n}}{b_{n}}$.
									\item $n<m$: the term $x^{m}$ will 'win', and drag the limit to $0$ (as in $\frac{1}{x}$, for example).
								\end{itemize}
							\end{answer}
						}\fi

					\item $\limit{x}{\pm\infty}\sin\left( x \right),\ \limit{x}{\pm\infty}\tan\left( x \right)$
						\if\withsol1{
								\begin{answer}
									Since $\sin\left( x \right)$ is periodic, there can be no limit when $x\rightarrow\pm\infty$: the function oscillates forever. The same is true for $\tan\left( x \right)$, except that the 'oscillation' in that case is between $-\infty$ and $+\infty$.
								\end{answer}
							}\fi

						\item $\limit{x}{ 0}\frac{\sin\left( x \right)}{x},\ \limit{x}{ 0}\sin\left( \frac{1}{x} \right)$
							\if\withsol1{
									\begin{answer}
										There are several equivalent ways to approach this, but we will look at one involving infinite sums. Thanks to Taylor expansions\footnotemark we know that the following is true:
										\begin{align*}
											\sin\left( x \right) &= \sum_{k=0}^{\infty} \frac{\left( -1 \right)^{k}}{2k+1}x^{2k+1}\\
											&= x - \frac{x^{3}}{3!} + \frac{x^{5}}{5!} - \frac{x^{7}}{7!} + \frac{x^{9}}{9!} - \dots
										\end{align*}
										We can see that all the terms like $x^{3},\ x^{5},\ x^{7},\ \dots$ approach $zero$ really fast, leaving $x$ dominant. Therefore, for a small $x$ $\sin\left( x \right)\approx x$, since both $\sin\left( x \right)$ and $x$ look similar. This means that $\limit{x}{ 0}\frac{\sin\left( x \right)}{x} = \limit{x}{ 0}\frac{x}{x} = 1$.\\

										The second limit, $\limit{x}{ 0}\sin\left( \frac{1}{x} \right)$ is not that simple. In short, since the smaller the $x$ the bigger $\frac{1}{x}$ is, we get that more of the function is being 'condensed' near $x=0$. Therefore, it oscillates faster and faster, and thus the limit at $0$ is undefined. See Figure \ref{fig:sine_one_over_x} for a graphical representation of this function.
										\begin{figure}[H]
											\centering
											\begin{tikzpicture}
												\begin{axis}[
														Axis Style,
														samples=500,
														xmin=-0.01, xmax=0.01,
														ymin=-1.5, ymax=1.5,
														domain=-0.01:0.01,
													]
													\addplot[function, thick, samples=500, col4] plot (\x, { sin(1/\x) });
												\end{axis}
											\end{tikzpicture} 
											\caption{The function $\sin\left( \frac{1}{x} \right)$ graphed for $x\in\left[ -0.01,0.01 \right]$.}
											\label{fig:sine_one_over_x}
										\end{figure}
									\end{answer}
									\footnotetext{Specifically, the McLaren series for $\sin\left( x \right)$. For further reading one should go to, of course, \href{https://en.wikipedia.org/wiki/Taylor_series}{Wikipedia}.}
								}\fi

							\item $\limit{x}{\pm\infty}e^{x},\ \limit{x}{\pm\infty}e^{-x},\ \limit{x}{0^{+}}\log\left( x \right),\ \limit{x}{\pm\infty}\log\left( x \right)$
								\if\withsol1{
										\begin{answer}
											Obviously, the bigger $x$ is, the bigger will $e^{x}$ be. Therefore
											\begin{equation*}
												\limit{x}{\infty}e^{x}=\infty.
											\end{equation*}

											On the other hand, when $x\rightarrow-\infty$, the expression $e^{x}$ gets smaller and smaller, since for any negative number $x=-a$, $e^{x}=\frac{1}{e^{a}}$, i.e. $1$ over a very large (positive) number. Mathematically speaking
											\begin{equation*}
												\limit{x}{-\infty}e^{x}=0.
											\end{equation*}

											For $f\left( x \right)=e^{-x}$ the exact opposite is true, since the expression $-x$ is an exact mirror of $x$ about the $y$-axis. Therefore,
											\begin{equation*}
												\limit{x}{\infty}e^{-x}=0
											\end{equation*}
											and
											\begin{equation*}
												\limit{x}{-\infty}e^{-x}=\infty.
											\end{equation*}
											See Figure \ref{fig:exponents} for a graph of both these functions.
											\begin{figure}[H]
												\centering
												\begin{tikzpicture}
													\Large
													\begin{axis}[
															Axis Style,
															xmin=-5, xmax=5,
															ymin=-5, ymax=5,
															domain=-5:5,
														]
														\addplot[function, col1!75] {exp(x)};
														\addplot[function, col2!75] {exp(-x)};
													\end{axis}
												\end{tikzpicture} 
												\caption{Graphing both $\color{col1!75}e^{x}$ and $\color{col2!75}e^{-x}$ for $x\in\left[ -3,3 \right]$.}
												\label{fig:exponents}
											\end{figure}

											Since $\log\left( x \right)$ is the inverse function of $e^{x}$, we can infer the behavior of $\log\left( x \right)$ from that of $e^{x}$. First, we recall that inverse functions behave as if their axes were swapped, which is equivalent to a $\ang{90}$ rotation followed by flipping the $y$ axis. In our case we get $\limit{x}{0^{+}}\log\left( x \right)=-\infty$ and $\limit{x}{ \infty}\log\left( x \right)=\infty$. See Figure \ref{fig:log} for a graphical representation of $\log\left( x \right)$.
											\begin{figure}[H]
												\centering
												\begin{tikzpicture}
													\Large
													\begin{axis}[
															Axis Style,
															xmin=-1, xmax=5,
															ymin=-10, ymax=2,
															domain=-1:5,
														]
														\addplot[function, samples=300, col3!75] {ln(x)};
													\end{axis}
												\end{tikzpicture} 
												\caption{Graph of $\log\left( x \right)$ for $x\in\left( 0, 5\right]$.}
												\label{fig:log}
											\end{figure}
										\end{answer}
									}\fi
\end{enumerate}


\chapter{Solutions}
\def\withsol{1}
\setcounter{ex}{1}
\forloop{ex}{1}{\value{ex} < 10}{
  \setcounter{probnum}{0}
  \def\exnum{\arabic{ex}}
  \rhead{Exercise \exnum}
  \include{ex\arabic{ex}}
  \newpage
}
\setcounter{probnum}{0}
\section{Limits of Real Functions}
Calculate the following limits:
\begin{enumerate}
	\item $\limit{x}{\pm\infty}x^{5}-3,\ \limit{x}{\pm\infty}4x^{3}-2x^{7}+103x^{5}$.
		\if\withsol1{
				\begin{answer}
					For a big enough $x$, $x^{5}\gg-3$ (the symbol $\gg$ means \textit{much bigger than}), and thus $\limit{x}{\infty}x^{5}-3=\infty$. The same argument works for a large negative $x$, although in this case it doesn't matter since we are subtracting $3$. Thus $\limit{x}{-\infty}x^{5}-3=-\infty$.\\

					Generally, when we are faced with a limit at $\pm\infty$ of a polynomial expression, the largest power of $x$ is the only thing that matters. Thus, for example, in $\limit{x}{\pm\infty}4x^{3}-2x^{7}+103x^{5}$ the only component of the polynomial that matters is $-2x^{7}$. Since $\limit{x}{\pm\infty}-2x^{7}=\mp\infty$ (i.e. when $x\rightarrow\infty,\ -2x^{7}\rightarrow-\infty$, and when $x\rightarrow-\infty,\ -2x^{7}\rightarrow\infty$), we get $\limit{x}{\pm\infty}4x^{3}-2x^{7}+103x^{5}=\mp\infty$.
				\end{answer}
			}\fi
		\item $\limit{x}{\pm\infty}\frac{1}{x},\ \limit{x}{ 0}\frac{1}{x},\ \limit{x}{\pm\infty}\frac{1}{x^{2}},\ \limit{x}{ 0}\frac{1}{x^{2}}$
			\if\withsol1{
					\begin{answer}
						The bigger the value of $x$, the smaller the expression $\frac{1}{x}$ gets. The same is of course true for large negative values of $x$, the only difference being that the values approach $0$ from the negative numbers. Thus, $\limit{x}{\pm\infty}\frac{1}{x}=0$. The opposite occurs for small values of $x$: as $x$ approaches $0$ the expression $\frac{1}{x}$ becomes larger and larger.

						However, the sign of $x$ plays a role, as $1$ over a positive number is positive, while $1$ over a negative number is negative. Thus when approaching $0$ from the positive numbers the limit tends towards $\infty$, while when approaching $0$ from the negative numbers the limits will go to $-\infty$. In mathematical notation:
						\begin{equation*}
							\limit{x}{0^{+}}\frac{1}{x}=\infty,\ \limit{x}{0^{-}}\frac{1}{x}=-\infty.
						\end{equation*}

						Therefore, the limit at $0$ just does not exist. For more insight, look at a graph of $f\left( x \right)=\frac{1}{x}$ (Figure \ref{fig:one_over_x}).
						\begin{figure}[H]
							\centering
							\begin{tikzpicture}
								\begin{axis}[
										Axis Style,
										xmin=-5, xmax=5,
										ymin=-5, ymax=5,
										domain=-5:5,
									]
									\addplot[function, col1!75] {1/x};
								\end{axis}
							\end{tikzpicture} 
							\caption{$f\left( x \right)=\frac{1}{x}$ for $x\in\left[ -5,5 \right]$.}
							\label{fig:one_over_x}
						\end{figure}

						The same analysis is true for $f\left( x \right)=\frac{1}{x^{2}}$ except that since $x^{2}$ is always non-negative, so is $\frac{1}{x^{2}}$, and thus the graph looks as in Figure \ref{fig:one_over_x_square} (notice also how $\frac{1}{x^{2}}$ diverges/decays much faster than $\frac{1}{x}$).

						Of course, since
						\begin{equation*}
							\limit{x}{0^{-}}\frac{1}{x^{2}}=\limit{x}{0^{+}}\frac{1}{x^{2}},
						\end{equation*}
						the limit at $0$ is defined and simply equals $0$.
						\begin{figure}[H]
							\centering
							\begin{tikzpicture}
								\begin{axis}[
										Axis Style,
										xmin=-5, xmax=5,
										ymin=-5, ymax=5,
										domain=-5:5,
									]
									\addplot[function, col2!75] {1/x^2};
								\end{axis}
							\end{tikzpicture} 
							\caption{$f\left( x \right)=\frac{1}{x^{2}}$ for $x\in\left[ -5,5 \right]$.}
							\label{fig:one_over_x_square}
						\end{figure}
					\end{answer}
				}\fi

			\item $\limit{x}{\infty}\frac{x^{4}-3x^{2}+10x}{-2x^{2}-5},\ \limit{x}{ -1}\frac{2x^{2}+x-1}{x+1}$
				\if\withsol1{
						\begin{answer}
							The highest power of $x$ in both the numerator and the denominator is $x^{4}$, so it controls the behavior of the function as $x\rightarrow\infty$. Since the coefficient of $x^{4}$ is positive (it is simply $1$), we get
							\begin{equation*}
								\limit{x}{\infty}\frac{x^{4}-3x^{2}+10x}{-2x^{2}-5}=\infty.
							\end{equation*}

							The other expression, $\frac{2x^{2}+x-1}{x+1}$, is somewhat tricky. If we pay close attention we can see that $2x^{2}+x-1=\left( 2x-1 \right)\left( x+1 \right)$, and thus
							\begin{align*}
								\frac{2x^{2}+x-1}{x+1} &= \frac{\left( 2x-1 \right)\cancel{\left( x+1 \right)}}{\cancel{x+1}}\\
								&= 2x-1.
							\end{align*}
							so we expect the two expressions to behave in the same way (i.e. be a simple line). However, $\frac{2x^{2}+x-1}{x+1}$ is still undefined at $x=-1$, and thus the function $f\left( x \right)=\frac{2x^{2}+x-1}{x+1}$ has a 'hole' at $x=-1$.

							Except for this point, however, it is well-behaved and looks exactly like $2x-1$, and thus 
							\begin{equation*}
								\limit{x}{ -1}\frac{2x^{2}+x-1}{x+1}=\limit{x}{ -1}2x-1=-3,
							\end{equation*}
							even though $\frac{2x^{2}+x-1}{x+1}$ is undefined at $x=-1$ (limits do not care about values at specific points! They only care about the behaviour leading to a point).
						\end{answer}
					}\fi

				\item $\limit{x}{\pm\infty}\frac{P_{n}\left( x \right)}{P_{m}\left( x \right)}$, where $P_{k}\left( x \right)$ is a real polynomial of order $k$, $n$ is even, $m$ is odd and $n>m$.\\
					\small{\underline{Note}: A real polynomial $P_{k}\left( x \right)$ is defined as $P_{k}\left( x \right)=\sum\limits_{i=0}^{k}a_{i}x^{i}$, with $a_{i}\in\mathbb{R}$ and $a_{k}\neq0$.}
					\if\withsol1{
							\begin{answer}
								Let's generalize what we saw in the previous paragraph: since for any polynomial the limit at $\pm\infty$ is depended only on the term with the highest power of $x$, we can write
								\begin{align*}
									\lim_{x\rightarrow\pm\infty}\frac{P_n\left( x \right)}{P_m\left( x \right)} &= \lim_{x\rightarrow\pm\infty}\frac{a_{n}x^{n}}{b_{n}x^{m}}\\
									&= \lim_{x\rightarrow\pm\infty}\frac{a_{n}}{b_{n}}x^{n-m}.
								\end{align*}
								where $a_{n}$ and $b_{n}$ are the coefficients of the terms $x^{n}$ and $x^{m}$ for $P_n\left( x \right)$ and $P_m\left( x \right)$, respectively.\\
								We can see that there are three possibilities:
								\begin{itemize}
									\item $n>m$: the limit would be $\pm\infty$, depending on the sign of $\frac{a_{n}}{b_{n}}$.
									\item $n=m$: in this case $x^{n-m}=x^{0}=1$, and thus the limit would be $\frac{a_{n}}{b_{n}}$.
									\item $n<m$: the term $x^{m}$ will 'win', and drag the limit to $0$ (as in $\frac{1}{x}$, for example).
								\end{itemize}
							\end{answer}
						}\fi

					\item $\limit{x}{\pm\infty}\sin\left( x \right),\ \limit{x}{\pm\infty}\tan\left( x \right)$
						\if\withsol1{
								\begin{answer}
									Since $\sin\left( x \right)$ is periodic, there can be no limit when $x\rightarrow\pm\infty$: the function oscillates forever. The same is true for $\tan\left( x \right)$, except that the 'oscillation' in that case is between $-\infty$ and $+\infty$.
								\end{answer}
							}\fi

						\item $\limit{x}{ 0}\frac{\sin\left( x \right)}{x},\ \limit{x}{ 0}\sin\left( \frac{1}{x} \right)$
							\if\withsol1{
									\begin{answer}
										There are several equivalent ways to approach this, but we will look at one involving infinite sums. Thanks to Taylor expansions\footnotemark we know that the following is true:
										\begin{align*}
											\sin\left( x \right) &= \sum_{k=0}^{\infty} \frac{\left( -1 \right)^{k}}{2k+1}x^{2k+1}\\
											&= x - \frac{x^{3}}{3!} + \frac{x^{5}}{5!} - \frac{x^{7}}{7!} + \frac{x^{9}}{9!} - \dots
										\end{align*}
										We can see that all the terms like $x^{3},\ x^{5},\ x^{7},\ \dots$ approach $zero$ really fast, leaving $x$ dominant. Therefore, for a small $x$ $\sin\left( x \right)\approx x$, since both $\sin\left( x \right)$ and $x$ look similar. This means that $\limit{x}{ 0}\frac{\sin\left( x \right)}{x} = \limit{x}{ 0}\frac{x}{x} = 1$.\\

										The second limit, $\limit{x}{ 0}\sin\left( \frac{1}{x} \right)$ is not that simple. In short, since the smaller the $x$ the bigger $\frac{1}{x}$ is, we get that more of the function is being 'condensed' near $x=0$. Therefore, it oscillates faster and faster, and thus the limit at $0$ is undefined. See Figure \ref{fig:sine_one_over_x} for a graphical representation of this function.
										\begin{figure}[H]
											\centering
											\begin{tikzpicture}
												\begin{axis}[
														Axis Style,
														samples=500,
														xmin=-0.01, xmax=0.01,
														ymin=-1.5, ymax=1.5,
														domain=-0.01:0.01,
													]
													\addplot[function, thick, samples=500, col4] plot (\x, { sin(1/\x) });
												\end{axis}
											\end{tikzpicture} 
											\caption{The function $\sin\left( \frac{1}{x} \right)$ graphed for $x\in\left[ -0.01,0.01 \right]$.}
											\label{fig:sine_one_over_x}
										\end{figure}
									\end{answer}
									\footnotetext{Specifically, the McLaren series for $\sin\left( x \right)$. For further reading one should go to, of course, \href{https://en.wikipedia.org/wiki/Taylor_series}{Wikipedia}.}
								}\fi

							\item $\limit{x}{\pm\infty}e^{x},\ \limit{x}{\pm\infty}e^{-x},\ \limit{x}{0^{+}}\log\left( x \right),\ \limit{x}{\pm\infty}\log\left( x \right)$
								\if\withsol1{
										\begin{answer}
											Obviously, the bigger $x$ is, the bigger will $e^{x}$ be. Therefore
											\begin{equation*}
												\limit{x}{\infty}e^{x}=\infty.
											\end{equation*}

											On the other hand, when $x\rightarrow-\infty$, the expression $e^{x}$ gets smaller and smaller, since for any negative number $x=-a$, $e^{x}=\frac{1}{e^{a}}$, i.e. $1$ over a very large (positive) number. Mathematically speaking
											\begin{equation*}
												\limit{x}{-\infty}e^{x}=0.
											\end{equation*}

											For $f\left( x \right)=e^{-x}$ the exact opposite is true, since the expression $-x$ is an exact mirror of $x$ about the $y$-axis. Therefore,
											\begin{equation*}
												\limit{x}{\infty}e^{-x}=0
											\end{equation*}
											and
											\begin{equation*}
												\limit{x}{-\infty}e^{-x}=\infty.
											\end{equation*}
											See Figure \ref{fig:exponents} for a graph of both these functions.
											\begin{figure}[H]
												\centering
												\begin{tikzpicture}
													\Large
													\begin{axis}[
															Axis Style,
															xmin=-5, xmax=5,
															ymin=-5, ymax=5,
															domain=-5:5,
														]
														\addplot[function, col1!75] {exp(x)};
														\addplot[function, col2!75] {exp(-x)};
													\end{axis}
												\end{tikzpicture} 
												\caption{Graphing both $\color{col1!75}e^{x}$ and $\color{col2!75}e^{-x}$ for $x\in\left[ -3,3 \right]$.}
												\label{fig:exponents}
											\end{figure}

											Since $\log\left( x \right)$ is the inverse function of $e^{x}$, we can infer the behavior of $\log\left( x \right)$ from that of $e^{x}$. First, we recall that inverse functions behave as if their axes were swapped, which is equivalent to a $\ang{90}$ rotation followed by flipping the $y$ axis. In our case we get $\limit{x}{0^{+}}\log\left( x \right)=-\infty$ and $\limit{x}{ \infty}\log\left( x \right)=\infty$. See Figure \ref{fig:log} for a graphical representation of $\log\left( x \right)$.
											\begin{figure}[H]
												\centering
												\begin{tikzpicture}
													\Large
													\begin{axis}[
															Axis Style,
															xmin=-1, xmax=5,
															ymin=-10, ymax=2,
															domain=-1:5,
														]
														\addplot[function, samples=300, col3!75] {ln(x)};
													\end{axis}
												\end{tikzpicture} 
												\caption{Graph of $\log\left( x \right)$ for $x\in\left( 0, 5\right]$.}
												\label{fig:log}
											\end{figure}
										\end{answer}
									}\fi
\end{enumerate}

\end{document}
  \if\specific1{
    \newcounter{probnum}
    \setcounter{probnum}{0}
    \def\exnum{\arabic{ex}}
    \rhead{Exercise \exnum}
    \def\withsol{0}
    \include{ex\arabic{ex}}
  }\fi

  \if\specific0{
