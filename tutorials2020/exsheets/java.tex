\documentclass[a4paper]{report}
\setlength{\parindent}{0pt}

\usepackage[explicit]{titlesec}
\usepackage[dotinlabels]{titletoc}
\usepackage[margin=2.0cm]{geometry}
\usepackage[bottom]{footmisc}
\usepackage{fancyhdr}
\usepackage{hyperref}
\usepackage{amsmath}
\usepackage{amssymb}
\usepackage{amsthm}
\usepackage{commath}
\usepackage{bm}
\usepackage[load={prefix}]{siunitx}
\usepackage{cancel}
\usepackage{xstring}
\usepackage{mathtools}
\usepackage{tabu}
\usepackage{enumitem}
\usepackage{float}
\usepackage{afterpage}
\newcommand\blankpage{%
	\null
	\thispagestyle{empty}%
	\addtocounter{page}{-1}%
\newpage}

\usepackage[table]{xcolor}
\definecolor{col1}{HTML}{FF7878}
\definecolor{col2}{HTML}{51B5F8}
\definecolor{col3}{HTML}{68E1AA}
\definecolor{col4}{HTML}{B869EA}
\definecolor{col5}{HTML}{FF5500}
\definecolor{col6}{HTML}{FF7878}
\definecolor{col7}{HTML}{FF7500}
\definecolor{col8}{HTML}{FF4F93}
\newcommand{\rcolor}[2]{
	\color{#1}{#2}\color{black}
}
\colorlet{colx}{col2!50}
\colorlet{coly}{col3!50}
\usepackage[most]{tcolorbox}
\usepackage{forloop}
\usepackage{listings}
\lstset{ 
    language=Java, % choose the language of the code
    basicstyle=\fontfamily{pcr}\selectfont\footnotesize\color{black},
    keywordstyle=\color{red}\bfseries, % style for keywords
    emphstyle=\color{col3}\bfseries,
    numbers=left, % where to put the line-numbers
    numberstyle=\color{blue}\tiny, % the size of the fonts that are used for the line-numbers
    numbersep=5pt,
    backgroundcolor=\color{white},
    showspaces=false, % show spaces adding particular underscores
    showstringspaces=false, % underline spaces within strings
    showtabs=false, % show tabs within strings adding particular underscores
    frame=single, % adds a frame around the code
    tabsize=2, % sets default tabsize to 2 spaces
		classoffset=1,
		otherkeywords={>,<,.,;,-,!,=,~},
		morekeywords={>,<,.,;,-,!,=,~},
		keywordstyle=\color{blue}\bfseries,
}
\newcommand{\LS}[1]{
	\mintinline{java}{#1}
}

\usepackage{pgf}
\usepackage{pgfplots}
\usepgfplotslibrary{fillbetween}
\pgfplotsset{every axis/.append style={
	font=\tiny,
	axis x line=middle,
	axis y line=middle,
	axis line style={<->, thick, black},
	xlabel={$x$},
	ylabel={$y$},
	every axis x label/.style={at={(current axis.right of origin)}, xshift=10pt, anchor=east},
	every axis y label/.style={at={(current axis.above origin)}, yshift=10pt, anchor=north},
	every extra x tick/.style={xticklabel style={above}},
	every extra y tick/.style={yticklabel style={right}}
}}

\usepackage{tikz}
\usetikzlibrary{tikzmark, calc, positioning, decorations.pathreplacing, shapes, backgrounds, arrows, automata, fit, turtle}
\tikzset{
	double arrow/.style args={#1 colored by #2 and #3}{
	-stealth,line width=#1,#2, % first arrow
	postaction={draw,-stealth,#3,line width=(#1)/3,
	shorten <=(#1)/3,shorten >=2*(#1)/3}, % second arrow
	}
}
\tikzset{vector/.style={->, >=stealth, very thick, cap=round}}

\newcommand{\tikznode}[3][]{%based on https://tex.stackexchange.com/a/402466/121799
\ifmmode%
\tikz[remember picture,baseline=(#2.base),inner sep=0pt] \node[#1] (#2) {$#3$};%
\else
\tikz[remember picture,baseline=(#2.base),inner sep=0pt] \node[#1] (#2) {#3};%
\fi}
\tikzset{
	My Node Style/.style={midway, right, xshift=3.0ex, align=left, font=\small, draw=none, thin, text=black},
	highlight/.style={rectangle, rounded corners, draw, fill opacity=0.5, inner sep=0pt},
	function/.style={ultra thick, samples=100, smooth}
}

\pgfkeys{/pgfplots/Axis Style/.style={
		axis on top=true,
		grid=major,
		axis x line=center,
		axis y line=middle,
		samples=100,
		xmin=-7.0, xmax=7.0,
		ymin=-1.5, ymax=1.5,
		label style={font=\large},
		x label style={at={(axis description cs:1,0.5)},anchor=west},
		y label style={at={(axis description cs:0.5,1)},anchor=south},
		tick label style={font=\small},
		axis line style={<->, stealth-stealth},
}}

\newcommand{\Highlight}[2][submatrix]{%
	\tikz[overlay,remember picture]{
	\node[highlight, fill=#2, fit=(left.north west) (right.south east)] (#1) {};}
}

% Norm of a vector
\newcommand\vnorm[2][black]{
	\left\| \color{#1} \vec{#2} \color{black} \right\|
}

\newcount\colveccount
\newcommand*\colvec[1]{
	\global\colveccount#1
	\begin{pmatrix}
	\colvecnext
	}
	\def\colvecnext#1{
	#1
	\global\advance\colveccount-1
	\ifnum\colveccount>0
	\\
	\expandafter\colvecnext
	\else
	\end{pmatrix}
	\fi
}

\newcommand\VerticalBrace[4][]{%
	% #1 = draw options
	% #2 = top mark
	% #3 = bottom mark
	% #4 = label
	\begin{tikzpicture}[overlay,remember picture]
	\draw[decorate,decoration={brace, amplitude=1.5ex}, #1] 
	([yshift=1ex]#2.north east)  -- ([yshift=-1ex]#3.south east)
	node[My Node Style] {#4};
	\end{tikzpicture}
}

\definecolor{light-gray}{gray}{0.95}

\newcommand{\dd}[2][]{
	\frac{\text{d}^{#1}}{\text{d}#2^{#1}}
}

\newcommand*{\rom}[1]{\expandafter\@slowromancap\romannumeral #1@}

\newcommand\rotMat[1]{%
	\begin{bmatrix}
	\cos\left( #1 \right) & -\sin\left( #1 \right) \\
	\sin\left( #1 \right) &  \cos\left( #1 \right) \\
	\end{bmatrix}
}

\DeclareMathOperator{\sgn}{sgn}

\newcommand{\xhl}[1][x]{
	\rcolor{col1}{#1}
}
\newcommand{\yhl}[1][y]{
	\rcolor{col2}{#1}
}
\newcommand{\zhl}[1][z]{
	\rcolor{col3}{#1}
}

\newcommand\digits{} % just for safety
\def\digits[#1](#2:#3:#4:#5)% [draw options](size:base:number)
{
	\pgfmathsetmacro{\n}{#5}
	\draw[#1] (0,0) rectangle (#2*\n,#2);
	\foreach \b in {1,...,\n}{
	\draw[#1,-] (#2*\b,0) to (#2*\b,#2);
	\pgfmathtruncatemacro{\A}{\n-\b}
	\node at (#2*\b-#2/2,-0.25) {$#3^{\A}$};
	\node at (#2*\b-#2/2,0.35) {$\StrChar{#4}{\b}$};
	}
}

\pgfdeclarelayer{background}
\pgfsetlayers{background,main}

\renewcommand\CancelColor{\color{red!70}}

\newcommand\issolution[1]{%
	\ifdefined\withsol
	#1
	\else
	\fi
}

\newcommand{\drawvec}[5]{%
	\coordinate (vec) at (#1,#2);
	\draw[-, dashed, gray] (#1,#2) to (0,#2);
	\draw[-, dashed, gray] (#1,#2) to (#1,0);
	\draw[->, #3, very thick] (0,0) to (vec);
	\node[#3, #4 = 0.01cm of vec] {\Large #5};
}

\newcommand\Rs[1]{\mathbb{R}^{#1}}

\newcommand{\highlight}[2][red]{\mathchoice%
	{\colorbox{#1}{$\displaystyle#2$}}%
	{\colorbox{#1}{$\textstyle#2$}}%
	{\colorbox{#1}{$\scriptstyle#2$}}%
{\colorbox{#1}{$\scriptscriptstyle#2$}}
}%

\newcommand{\solutionhead}{
	\if\withsol0{
	}\fi
	\if\withsol1{
	(Solution)
	}\fi
}

\newenvironment{answer}
	{
	\begin{tcolorbox}[breakable, enhanced]
	\textbf{Answer:}\\
	}
	{
	\end{tcolorbox}
	}

\pagestyle{fancy}
\lhead{Computer Science and Mathematics (B.MES.108), SoSe20}

\title{Computer Science and Mathematics (B.MES.108)\\Tutorial, Summer Semester 2020\\}
\author{Peleg Bar Sapir}
 
\rhead{Exercise Sheet: Java\solutionhead}

\newcommand{\lsys}[5]{
  \begin{tabular}[H]{ll}
    \textbf{variables}:& #1 \\
    \textbf{constants}:& #2 \\
    \textbf{axiom}:& #3 \\
    \textbf{rules}:& #4 \\
    \textbf{$N$}:& #5
  \end{tabular}
}

\renewcommand{\thesubsection}{Problem \arabic{subsection}}

\usepackage{minted}
\setminted[java]{
	frame=lines,
	framesep=2mm,
	baselinestretch=1,
	bgcolor=lightgray!10!white,
	linenos,
	xleftmargin=4mm,
	breaklines,
	firstnumber=1,
}

\begin{document}
\section*{Extra Java Problems\solutionhead}
\subsection{}
	The following function is defined:
	\inputminted[firstline=38, lastline=47]{java}{../java/code.java}
	\textbf{Note}: the line
	\inputminted[firstline=41, lastline=41, firstnumber=4]{java}{../java/code.java}
		creates a new array of integers of length \LS{n} called \LS{arr2}.
	\begin{enumerate}
		\item What will the function return for the input \LS{[1,0,5,7,2,3]}?
			\if\withsol1{
				\begin{answer}
					First, line \LS{3} will assign the variable \LS{n} a value of \LS{6} (since the length of the array is $6$).

					Then, line \LS{4} creates a new array called \LS{arr2} of length \LS{n=6}.

					Let's now follow the \LS{for} loop using a table:\\
					
					\centering
						\begin{tabular}{c|c|c}
							\hline
							\LS{i} & \LS{n-i-1} & \LS{arr[n-i-1]}\\
							\hline
							0 & 5 & 3 \\
							1 & 4 & 2 \\
							2 & 3 & 7 \\
							3 & 2 & 5 \\
							4 & 1 & 0 \\
							5 & 0 & 1 \\
							\hline
						\end{tabular}~\\

					\flushleft
					Meaning that for the input \LS{[1,0,5,7,2,3]} the function \LS{Q1} returns \LS{[3,2,7,5,0,1]}.
				\end{answer}
			}\fi

		\item What does the function return in general?
			\if\withsol1{
				\begin{answer}
					The function returns a reversed copy of the input array.
				\end{answer}
			}\fi
	\end{enumerate}

\subsection{}
	The following function is defined:
	\inputminted[firstline=49, lastline=64]{java}{../java/code.java}
	Follow the code carefully and explain what does the function do.
		\if\withsol1{
			\begin{answer}
				The lines
				\inputminted[firstline=51, lastline=52]{java}{../java/code.java}
				checks whether the two input arrays have identical lengths. If they don't, then it immediatly returns \LS{false}.
				~\\

				If the two input arrays have the same lengths, the function continues to create two variables, a boolean called \LS{condition} which is set to \LS{true}, and an integer \LS{i} set to \LS{0}. A \LS{while} loop then runs so long as \LS{condition} is \LS{true} and \LS{i} is smaller then the length of \LS{arr1} (which is also the length of array \LS{arr2}). If any of the elements \LS{i} of the two arrays are not identical, \LS{condition} is set to \LS{false} and the \LS{while} loop stops. Otherwise it continues to interate with the integer \LS{i} until it reaches the length of \LS{arr1}.

				~\\
				The function then returns the value of \LS{condition}.

				~\\
				What the function does is to check whether the two input arrays are identical. If they are, it returns \LS{true}, otherwise it returns \LS{false}.
			\end{answer}
		}\fi

\subsection{}
The Fibonacci sequence $\left\{ F_{i} \right\}$ is defined as follows:
\begin{equation*}
	F_{i} = F_{i-2} + F_{i-1},
\end{equation*}
with $F_{0} = F_{1} = 1$.

For example, the first $10$ terms of the sequence are
\begin{equation*}
	1, 1, 2, 3, 5, 8, 13, 21, 34, 55, \dots
\end{equation*}

\begin{enumerate}
	\item Write a Java function that takes an integer $n\geq2$ and returns the first $n$ values of the Fibonacci sequence.
		\if\withsol1{
			\begin{answer}
				\inputminted[firstline=66, lastline=75]{java}{../java/code.java}
			\end{answer}
		}\fi

	\item \textbf{Challange}: Write a \textbf{recursive} function that takes an integer $n\geq0$ and returns the value of $F_{n}$.
		\if\withsol1{
			\begin{answer}
				\inputminted[firstline=77, lastline=82]{java}{../java/code.java}
			\end{answer}
		}\fi
\end{enumerate}
\end{document}
