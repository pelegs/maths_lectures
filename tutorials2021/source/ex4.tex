\section{Determinants, Systems of Linear Equations}

\subsection{Determinants}
\begin{enumerate}
	\item Find the determinants of the following matrices:
		\begin{enumerate}
			\item $\begin{pmatrix} 1 & 3 \\ 2 & 0 \\ \end{pmatrix}$
				\if\withsol1{
						\begin{answer}
							$\det\left( \bm{M} \right)=\cancel{1\cdot0}-2\cdot3=-6$.
					\end{answer}}
				\fi

			\item $\begin{pmatrix} 1 & 3 & -2 \\ 2 & 0 & 7 \\ 4 & 5 & 5 \end{pmatrix}$
				\if\withsol1{
						\begin{answer}
							A quick reminder: a minor $m_{ij}$ of a matrix $\bm{M}$ is the matrix resulting from removing the row $i$ and column $j$ from the matrix.

							Let us start with element $a_{11}$ (which equals $1$ in this case); we 'hide' the row and column of this element, and calculate the determinant of the resulting minor:
							\begin{equation*}
								\bm{M}=\left(
									\begin{array}{>{\columncolor{black}}ccc}
										\rowcolor{black}
										1 & 3 & -2 \\
										2 & 0 & 7  \\
										4 & 5 & 5  \\
									\end{array}
								\right) \longrightarrow m_{11}=
								\begin{pmatrix}
									0 & 7 \\
									5 & 5 \\
								\end{pmatrix} \longrightarrow
								\det\left( m_{11} \right)=\cancel{0\cdot5}-5\cdot7=-35.
							\end{equation*}

							We continue to the next element of the first row ($a_{12}=3$) and do the same:
							\begin{equation*}
								\bm{M}=\left(
									\begin{array}{c>{\columncolor{black}}cc}
										\rowcolor{black}
										1 & 3 & -2 \\
										2 & 0 & 7  \\
										4 & 5 & 5  \\
									\end{array}
								\right) \longrightarrow m_{12}=
								\begin{pmatrix}
									2 & 7 \\
									4 & 5 \\
								\end{pmatrix} \longrightarrow
								\det\left( m_{12} \right)=2\cdot5-4\cdot7=-18.
							\end{equation*}

							And for $a_{13}=-2$:
							\begin{equation*}
								\bm{M}=\left(
									\begin{array}{cc>{\columncolor{black}}c}
										\rowcolor{black}
										1 & 3 & -2 \\
										2 & 0 & 7  \\
										4 & 5 & 5  \\
									\end{array}
								\right) \longrightarrow m_{13}=
								\begin{pmatrix}
									2 & 0 \\
									4 & 5 \\
								\end{pmatrix} \longrightarrow
								\det\left( m_{13} \right)=2\cdot5-\cancel{4\cdot0}=10.
							\end{equation*}

							The resulting determinant is thus:
							\begin{align*}
								\det\left( \bm{M} \right)&=
								a_{11}\cdot\det\left( m_{11} \right)-a_{12}\cdot\det\left( m_{12} \right)+a_{13}\cdot\det\left( m_{13} \right)\\
								&=1\left(-35\right) -3\left(-18\right) -2\left(10\right)\\
								&=-35 + 54 - 20\\
								&=-1.
							\end{align*}
					\end{answer}}
				\fi

			\item $\begin{pmatrix} 3 & 0 & 0 \\ 0 & 3 & 0 \\ 0 & 0 & 3 \end{pmatrix}$
				\if\withsol1{
						\begin{answer}
							\underline{The easy way}: if $\bm{A}$ is a matrix of dimension $n\times n$ then $\det\left( \alpha\cdot \bm{A} \right)=\alpha^{n}\cdot\det\left( \bm{A} \right)$.\\ In our case, $\bm{M}=3\cdot \bm{I}_{3}$, and therefore $\det\left( \bm{M} \right)=3^{3}\cdot\det\left( \bm{I}_{3} \right)=3^{3}=27$.
					\end{answer}}
				\fi

			\item $
				\begin{pmatrix}
					1 & 0 & 2 \\
					0 & 0 & 0 \\
					2 & 4 & 7 \\
				\end{pmatrix}$
				\if\withsol1{

						\begin{answer}
							\underline{The easy way}: any matrix that has one or more zero-rows (or zero-columns) has a determinant of $0$.
					\end{answer}}
				\fi

			\item $
				\begin{pmatrix}
					4 & 2 & 2 \\
					2 & 1 & 1 \\
					2 & 0 & 7 \\
				\end{pmatrix}$
				\if\withsol1{

						\begin{answer}
							\underline{The easy way}: notice that $\left( 4, 2, 2 \right)=2\cdot\left( 2, 1, 1 \right)$, meaning that these rows are linearly depended (and therefore the rank of the matrix is smaller than its row dimension). Any such matrix has a determinant of $0$.
					\end{answer}}
				\fi
		\end{enumerate}
	\item What does it mean when the determinant of a $2\times2$ matrix is negative?
		\if\withsol1{

				\begin{answer}
					It means that the operation the matrix performs flips the space, either vertically or horizontally (but not both).
				\end{answer}
			}\fi
	\end{enumerate}

	\subsection{Row Operations and Rank}
	A matrix is said to be in its \textit{Row Echelon Form} if:
	\begin{enumerate}
		\item All nonzero rows (rows with at least one nonzero element) are above any rows of all zeroes (all zero rows, if any, belong at the bottom of the matrix)
		\item The leading coefficient (the first nonzero number from the left) of a nonzero row is always strictly to the right of the leading coefficient of the row above it.
	\end{enumerate}

	If the leading coefficients are all $1$, then the form is called a \textit{Reduced Row Echelon Form}.\\
	For example, the following matrices are all presented in reduced row echelon form:
	\begin{align*}
		\begin{pmatrix}
			1 & -2 & 0 & 5\\
			0 & 1  & 3 & 7\\
			0 & 0  & 0 & 1\\
			0 & 0  & 0 & 0\\
		\end{pmatrix},\quad
		\begin{pmatrix}
			1 & 7\\
			0 & 1\\
		\end{pmatrix},\quad
		\begin{pmatrix}
			1 & 3 & 7\\
			0 & 1 & 2\\
		\end{pmatrix},\quad
		\begin{pmatrix}
			1 & 0 & 5\\
			0 & 1 & 1\\
			0 & 0 & 1\\
			0 & 0 & 0\\
			0 & 0 & 0\\
		\end{pmatrix}.
	\end{align*}

	\begin{enumerate}
		\item Use row operations to bring the following two matrices to their reduced row echelon form:
			\begin{equation*}
				A=
				\begin{pmatrix}
					3 & 1 & 0\\
					6 & 2 & 1\\
					3 & 6 & 1\\
				\end{pmatrix},\quad
				B=
				\begin{pmatrix}
					1  & 0 & 1\\
					2  & 0 & 2\\
					-3 & 0 & -3\\
				\end{pmatrix}.
				\end{equation*}

					\if\withsol1{

							\begin{answer}
								\begin{align*}
									&\begin{pmatrix}
										3 & 1 & 0\\
										6 & 2 & 1\\
										3 & 6 & 1
									\end{pmatrix}\xrightarrow[]{R_{2}\rightarrow R_{2}-2R_{1}}
									\begin{pmatrix}
										3 & 1 & 0\\
										0 & 0 & 1\\
										3 & 6 & 1
									\end{pmatrix}\xrightarrow[]{R_{2}\leftrightarrow R_{3}}
									\begin{pmatrix}
										3 & 1 & 0\\
										3 & 6 & 1\\
										0 & 0 & 1
									\end{pmatrix}\xrightarrow[]{R_{1}\rightarrow R_{1}-R_{2}}
									\begin{pmatrix}
										0 & -5 & -1\\
										3 & 6 & 1\\
										0 & 0 & 1
									\end{pmatrix}\xrightarrow[]{R_{1}\rightarrow R_{1}+R_{3}}\\[5mm]
									&\begin{pmatrix}
										0 & -5 & 0\\
										3 & 6 & 1\\
										0 & 0 & 1
									\end{pmatrix}\xrightarrow[]{R_{1}\rightarrow-\frac{1}{5}R_{1}}
									\begin{pmatrix}
										0 & 1 & 0\\
										3 & 6 & 1\\
										0 & 0 & 1
									\end{pmatrix}\xrightarrow[]{R_{1}\leftrightarrow R_{2}}
									\begin{pmatrix}
										3 & 6 & 1\\
										0 & 1 & 0\\
										0 & 0 & 1
									\end{pmatrix}\xrightarrow[]{R_{1}\rightarrow R_{1}-6R_{2}}
									\begin{pmatrix}
										3 & 0 & 1\\
										0 & 1 & 0\\
										0 & 0 & 1
									\end{pmatrix}\xrightarrow[]{R_{1}\rightarrow R_{1}-R_{3}}\\[5mm]
									&\begin{pmatrix}
										3 & 0 & 0\\
										0 & 1 & 0\\
										0 & 0 & 1
									\end{pmatrix}\xrightarrow[]{R_{1}\rightarrow\frac{1}{3}R_{1}}
									\begin{pmatrix}
										1 & 0 & 0\\
										0 & 1 & 0\\
										0 & 0 & 1
									\end{pmatrix}.
								\end{align*}


								~\\~\\
								\begin{align*}
									\begin{pmatrix}
										1 & 0 & 1\\
										2 & 0 & 2\\
										-3 & 0 & -3
									\end{pmatrix}\xrightarrow[]{R_{2}\rightarrow R_{2}-2R_{1}}
									\begin{pmatrix}
										1 & 0 & 1\\
										0 & 0 & 0\\
										-3 & 0 & -3
									\end{pmatrix}\xrightarrow[]{R_{3}\rightarrow R_{3}+3R_{1}}
									\begin{pmatrix}
										1 & 0 & 1\\
										0 & 0 & 0\\
										0 & 0 & 0
									\end{pmatrix}.
								\end{align*}
							\end{answer}
						}\fi

					\item What are the ranks of $A$ and $B$?
						\if\withsol1{

								\begin{answer}
									Row operations do not change the rank of matrices, and so: $\text{rank}\left( A \right)=3,\ \text{rank}\left( B \right)=1$.
								\end{answer}
							}\fi
					\end{enumerate}

					\subsection{Systems of Linear Equations}
					The following system of linear equations is given:
					\begin{equation*}
						\begin{cases}
							-x+3z = 20\\
							3x+y+3z = 15\\
							9x+3y = -18\\
						\end{cases}.
					\end{equation*}
					Solve the system using the Gaussian elimination method.

					\if\withsol1{
							\begin{answer}
								First, let us write the system in matrix-vector form:
								\begin{equation*}
									\begin{pmatrix}
										-1 & 0 & 3 \\
										3  & 1 & 3 \\
										9  & 3 & 0 \\
									\end{pmatrix}\colvec{3}{x}{y}{z} = \colvec{3}{20}{15}{-18}.
								\end{equation*}

								We will perform the Gaussian elimination process on the augmented matrix
								\begin{equation*}
									\left(\begin{array}{ccc|c}
											-1 & 0 & 3 & 20\\
											3  & 1 & 3 & 15\\
											9  & 3 & 0 & -18
									\end{array}\right).
								\end{equation*}

								\begin{align*}
									&
									\left(\begin{array}{ccc|c}
											-1 & 0 & 3 & 20\\
											3  & 1 & 3 & 15\\
											9  & 3 & 0 & -18
									\end{array}\right) \xrightarrow[] {-\frac{1}{R_{1}} \rightarrow R_{1}}
									\left(\begin{array}{ccc|c}
											1 & 0 & -3 & -20\\
											3  & 1 & 3 & 15\\
											9  & 3 & 0 & -18
									\end{array}\right) \xrightarrow[] {R_{2}-3R_{1} \rightarrow R_{2}}\\
									&
									\left(\begin{array}{ccc|c}
											1 & 0 & -3 & -20\\
											0  & 1 & 12 & 75\\
											9  & 3 & 0 & -18
									\end{array}\right) \xrightarrow[] {R_{3}-9R_{1} \rightarrow R_{3}}
									\left(\begin{array}{ccc|c}
											1 & 0 & -3 & -20\\
											0  & 1 & 12 & 75\\
											0  & 3 & 27 & 162
									\end{array}\right) \xrightarrow[] {R_{3}-3R_{2} \rightarrow R_{3}}\\
									&
									\left(\begin{array}{ccc|c}
											1 & 0 & -3 & -20\\
											0  & 1 & 12 & 75\\
											0  & 0 & -9 & -63 
									\end{array}\right) \xrightarrow[] {-\frac{1}{9}R_{3} \rightarrow R_{3}}
									\left(\begin{array}{ccc|c}
											1 & 0 & -3 & -20\\
											0  & 1 & 12 & 75\\
											0  & 0 & 1 & 7 
									\end{array}\right) \xrightarrow[] {R_{1}+3R_{3} \rightarrow R_{1}}\\
									&
									\left(\begin{array}{ccc|c}
											1 & 0 & 0 & 1\\
											0  & 1 & 12 & 75\\
											0  & 0 & 1 & 7 
									\end{array}\right) \xrightarrow[] {R_{2}-12R_{3} \rightarrow R_{2}}
									\left(\begin{array}{ccc|c}
											1 & 0 & 0 & 1\\
											0  & 1 & 0 & -9\\
											0  & 0 & 1 & 7 
									\end{array}\right).
								\end{align*}

								Thus, the solution to the system is
								\begin{equation*}
									x=1,\quad y=-9,\quad z=7.
								\end{equation*}
							\end{answer}
						}\fi
