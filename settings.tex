%% ----- General Packages ----- %%

\usepackage{pifont, mdframed}
\usepackage{hyphenat}
\usepackage{hyperref}
\usepackage{soul}
\usepackage{float}
\usepackage{needspace}
\usepackage{pythontex}
\usepackage{tabularx}
\usepackage{colortbl}
\usepackage{booktabs}
\usepackage{xargs}
\usepackage{xstring}

%% ----- Graphics ----- %%

\usepackage{caption}
\usepackage{subcaption}
\usepackage{svg}
\definecolor{col0}{HTML}{FFFFFF}
\definecolor{col1}{HTML}{FF7878}
\definecolor{col2}{HTML}{51B5F8}
\definecolor{col3}{HTML}{68E1AA}
\definecolor{col4}{HTML}{B869EA}
\definecolor{col5}{HTML}{FF5500}
\definecolor{col6}{HTML}{FF7878}
\definecolor{col7}{HTML}{FF7500}
\definecolor{col8}{HTML}{FF4F93}
\graphicspath{{./figures/}}
\newcommand{\rcolor}[2]{
  \color{#1}{#2}\color{black}
}

%% ----- Maths ----- %%

\usepackage{amsmath}
\usepackage{amssymb}
\usepackage{amsthm}
\usepackage{amsfonts}
\usepackage{bm}
\usepackage{commath}
\usepackage{gensymb}
\usepackage{scalerel}
\usepackage[thicklines]{cancel}
\renewcommand\CancelColor{\color{col1}}
\usepackage{mathtools}
\usepackage{siunitx}
\usepackage{nicematrix}
\allowdisplaybreaks
\newcommand{\defeq}{\vcentcolon=}
\newcommand{\eqdef}{=\vcentcolon}
\newcommand{\comp}{\mathsf{c}}
\newcommand\Rs[1]{\mathbb{R}^{#1}}
\newcommand\eb[1]{\hat{e}_{#1}}
\newcommand\pdot[2]{\vec{#1}\cdot\vec{#2}}
\newcommand\pcross[2]{\vec{#1}\times\vec{#2}}
\newcommand\func[3]{#1:#2\rightarrow#3}

% test
\newcommand{\pgfprint}[1]{
  \pgfmathparse{#1}\pgfmathprintnumber\pgfmathresult
}

% Empty square
\def\msquare{\mathord{\scalerel*{\Box}{gX}}}

% Column vectors
\newcount\colveccount
\newcommand*\colvec[1]{
  \global\colveccount#1
  \begin{pmatrix}
  \colvecnext
  }
  \def\colvecnext#1{
  #1
  \global\advance\colveccount-1
  \ifnum\colveccount>0
  \\
  \expandafter\colvecnext
  \else
  \end{pmatrix}
  \fi
}

% Dim
\renewcommand{\dim}{
  \operatorname{dim}
}

% Rank
\newcommand{\rank}{
  \operatorname{rank}
}

% Null
\newcommand{\nullspace}{
  \operatorname{Null}
}

% Trace
\newcommand{\tr}{
  \operatorname{tr}
}

% Distance
\newcommand{\dist}{
  \operatorname{dist}
}

% Identity matrix
\newcommand\idenmat[1]{
  \pyc{import sys}
  \pyc{sys.path.append('scripts')}
  \pyc{from scripts import identity_matrix}
  \pyc{identity_matrix(#1)}
}

% General identity matrix 1
\newcommand\idnmtrx{
  \NiceMatrixOptions{xdots/shorten = 0.6 em}
  \begin{pNiceMatrix}
  1& 0& \Cdots  &0     \\
  0& 1& \Ddots  &\Vdots\\
  \Vdots  &\Ddots  & 1&0\\
  0&\Cdots  & 0&1
  \CodeAfter
  \line{2-2}{3-3}
  \end{pNiceMatrix}
}

% General identity matrix 2
\newcommand{\genidmt}{
  \begin{psmallmatrix}
  1 & 0 & 0 & \dots & 0 \\
  0 & 1 & 0 & \dots & 0 \\
  0 & 0 & 1 & \dots & 0 \\
  \vdots & \vdots & \vdots & \ddots & \vdots \\
  0 & 0 & 0 & \dots & 1 \\
  \end{psmallmatrix}
}

% Norm of a vector
\newcommand\vnorm[2][black]{
  \left\| \color{#1} \vec{#2} \color{black} \right\|
}

% Inner product
\newcommand\dotp[2]{
  \langle #1,#2 \rangle
}

% Reflection and Rotation matrices
\DeclareMathOperator{\Reflect}{Ref}
\DeclareMathOperator{\Rot}{Rot}

% Actual matrices
\newcommandx*{\RotMat}[4][1=\theta, 2=\theta, 3=\theta, 4=\theta]{
  \begin{pmatrix}
  \cos(#1) & -\sin(#2)\\
  \sin(#3) & \cos(#4)
  \end{pmatrix}
}

% Projection of vectors
\newcommandx*{\projection}[4][1=u, 2=v, 3=col1, 4=col2]{
  \operatorname{proj}_{\rcolor{#4}{\vec{#2}}}\rcolor{#3}{\vec{#1}}
}

% Axes
\newcommand{\xaxis}[2]{
  \draw[vector, <->] (#1,0) -- (#2,0) node [right] {$x$};
}
\newcommand{\yaxis}[2]{
  \draw[vector, <->] (0,#1) -- (0,#2) node [above] {$y$};
}

%% ----- Bibliography ----- %%

\usepackage[backend=biber, style=nature]{biblatex}
\bibliography{./bib/ref}

%% ----- TikZ and PGF ----- %%

\usepackage{tikz}
\usepackage{tikz-3dplot}
\usepackage{xspace}
\newcommand{\TikZ}{\textup{Ti\textit{k}Z}}
\usepackage[customcolors]{hf-tikz}
\usetikzlibrary{positioning, calc, math, norm, shapes, fit, backgrounds, arrows, automata, decorations.pathreplacing, decorations.text, trees, matrix}
\newcommand\tikzmark[1]{\tikz[overlay,remember picture] \node (#1) {};}
\tikzset{
  double arrow/.style args={#1 colored by #2 and #3}{
  -stealth,line width=#1,#2, % first arrow
  postaction={draw,-stealth,#3,line width=(#1)/3,
  shorten <=(#1)/3,shorten >=2*(#1)/3}, % second arrow
},
	declare function={gauss(\x,\y,\z) = exp(-(\x-\y)^2/(2*\z^2));},
	declare function={func(\x) = gauss(\x, \mu, \sig);},
	declare function={gaussderiv(\x,\y,\z) = -(\x-\y)*gauss(\x,\y,\z)/\z^2;},
	declare function={funcderiv(\x) = gaussderiv(\x, \mu, \sig);},
}

\newcommand{\tikznode}[3][]{%based on https://tex.stackexchange.com/a/402466/121799
  \ifmmode%
  \tikz[remember picture,baseline=(#2.base),inner sep=0pt] \node[#1] (#2) {$#3$};%
  \else
  \tikz[remember picture,baseline=(#2.base),inner sep=0pt] \node[#1] (#2) {#3};%
  \fi
}
\tikzstyle{every edge}=[draw=black, very thick]
\tikzset{
  My Node Style/.style={midway, right, xshift=3.0ex, align=left, font=\small, draw=none, thin, text=black},
  highlight/.style={rectangle, rounded corners, draw, fill opacity=0.5, inner sep=0pt},
  vector/.style={->, >=stealth, very thick},
  perp/.style={-, densely dotted, thick, black!50},
  graphstate/.style={draw=black, very thick, text=black, minimum size=7mm, inner sep=0pt, outer sep=0pt},
  treenode/.style = {align=center, inner sep=0pt, text centered,
  font=\sffamily},
  tnode/.style = {treenode, circle, font=\sffamily\bfseries, draw=black,
  fill=col2!50, text width=1.5em},
  rnode/.style = {treenode, circle, font=\sffamily\bfseries, draw=black,
  fill=col1!50, text width=1.5em},
  lnode/.style = {treenode, circle, font=\sffamily\bfseries, draw=black,
  fill=col3!50, text width=1.5em},
  arr/.style = {vector, densely dotted},
  annotation/.style = {draw=black!60},
  annotationtext/.style = {text=black!60},
}
\def\dis{1.0cm}

\newcommand{\boxgrid}[6]{
  \pgfmathsetmacro{\a}{#1}
  \pgfmathsetmacro{\nrow}{#2}
  \pgfmathsetmacro{\ncol}{#3}
  \pgfmathsetmacro{\dx}{#4}
  \pgfmathsetmacro{\dy}{#5}
  \foreach \row in {1,...,\nrow}{
  \foreach \col in {1,...,\ncol}{
  \draw[thick] (\row*\a+\dx,-\col*\a-\dy) rectangle ++(\a,-\a) node [midway] {$#6_{\col\row}$};
  }
  }
  \draw[thick] (\dx+\a*1.7,\dy-\a*0.7) -- ++(-\a*2,0);
}

%% ----- PGF Plots stuff ----- %%

\usepackage{pgf}
\pgfdeclarelayer{background layer}
\pgfdeclarelayer{foreground layer}
\pgfsetlayers{background layer,main,foreground layer}
\usepackage{pgfplots}
\pgfplotsset{compat=1.16}
\usepgfplotslibrary{fillbetween, colormaps, colorbrewer}
\pgfplotsset{
	every axis/.append style={
  font=\small,
  axis x line=middle,
  axis y line=middle,
  axis line style={<->, thick, black},
  xlabel={$x$},
  ylabel={$y$},
  every axis x label/.style={at={(current axis.right of origin)}, xshift=10pt, anchor=east},
  every axis y label/.style={at={(current axis.above origin)}, yshift=10pt, anchor=north},
  every extra x tick/.style={xticklabel style={above}},
  every extra y tick/.style={yticklabel style={right}}
}}

\pgfkeys{/pgfplots/Axis Style/.style={
    axis x line=center,
    axis y line=middle,
    samples=200,
    xmin=-7, xmax=7,
    ymin=-7, ymax=7,
    axis line style={<->, stealth-stealth},
		%axis y line=none,
		ticks=none,
}}

\pgfplotsset{
  funcdraw/.style={thick},
}

\newcommandx*{\drawaxes}[9][5=1, 6=black!25, 7=densely dotted, 8=black, 9=1]{
  \ifthenelse{#9=0}
  {}
  {
  \foreach \x in {#1,...,#3}{
  \ifthenelse{\x = 0}
  { }
  {\draw[-, #8] (\x,-0.1) -- (\x,0.1) node [below, yshift=-1mm] {\tiny$\x$};}
  }
  \foreach \y in {#2,...,#4}{
  \ifthenelse{\y = 0}
  { }
  {\draw[-, #8] (-0.1,\y) -- (0.1,\y) node [left, xshift=-1mm] {\tiny$\y$};}
  }
  }
  \draw[step=#5, #6, thin, #7] (#1,#2) grid (#3,#4);
  \draw[vector, thick, <->] (#1-0.5,0) to (#3+0.5,0) node [right] {$x$}; 
  \draw[vector, thick, <->] (0,#2-0.5) to (0,#4+0.5) node [above] {$y$};
}
  
\newcommandx*{\boxmatrix}[9][1=5, 2=7, 3=0.5, 4=0.5, 5=0, 6=0, 7=1, 8=M, 9=N, usedefault]{
  \pgfmathsetmacro{\Nx}{#1}
  \pgfmathsetmacro{\Ny}{#2}
  \pgfmathsetmacro{\Lx}{#3}
  \pgfmathsetmacro{\Ly}{#4}
  \pgfmathsetmacro{\width}{\Lx*\Nx}
  \pgfmathsetmacro{\height}{\Ly*\Ny}
  \pgfmathsetmacro{\px}{#5}
  \pgfmathsetmacro{\py}{#6}
  \coordinate (corner1) at ($(\px,\py)-(\width*0.5,\height*0.5)$);
  \coordinate (corner2) at ($(\px,\py)+(\width*0.5,\height*0.5)$);

  \filldraw[col#7!50, draw=col#7, thick] (corner1) rectangle (corner2);
  \foreach \col in {1,...,\Nx}
  \draw[-, thick, col#7] ($(corner1)+(\col*\Lx,0)$) -- ++($(0, \height)$);
  \foreach \row in {1,...,\Ny}
  \draw[-, thick, col#7] ($(corner1)+(0,\row*\Ly)$) -- ++($(\width, 0)$);
  \draw [col#7!75!black, thick, cap=round, decorate, decoration={brace, amplitude=3pt, raise=4pt}]
  (corner1) -- ++(0,\height) node[midway, xshift=-15pt, text=col#7!75!black]{$#8$};
  \draw [col#7!75!black, thick, cap=round, decorate, decoration={brace, amplitude=3pt, raise=4pt, mirror}]
  (corner2) -- ++(-\width,0) node[midway, yshift=15pt, text=col#7!75!black]{$#9$};

}

\newcommandx*{\genvec}[2][1=x, 2=fff]{
  \IfEq{#2}{fff}{
  \colvec{4}{#1_{1}}{#1_{2}}{\vdots}{#1_{n}}
  }{
  \colvec{4}{#1_{1}+#2_{1}}{#1_{2}+#2_{2}}{\vdots}{#1_{n}+#2_{n}}
  }
}

\newcommand{\minorblack}[1]{
  \tikzmarkin[set fill color=black, set border color=black]{#1}
}

\newcommand{\xhat}{
  \draw[vector, col1] (0,0) -- (1,0) node [above] {$\hat{x}$};
}
\newcommand{\yhat}{
  \draw[vector, col2] (0,0) -- (0,1) node [right] {$\hat{y}$};
}

%% ----- Math definition ----- %%

\mdfdefinestyle{theoremstyle}
{
  linewidth=2pt,
  frametitlerule=true,
  innertopmargin=\topskip,
}
\mdtheorem[
  style=theoremstyle,
  linecolor=col4!75,
  frametitlebackgroundcolor=col4!25
  ]{mathdef}{\color{col4!25!black}Definition}

%% ----- Attention/Warning environment ----- %%

\mdfdefinestyle{theoremstyle}
{
  linewidth=2pt,
  frametitlerule=true,
  innertopmargin=\topskip,
}
\mdtheorem[
  style=theoremstyle,
  linecolor=col1!75,
  frametitlebackgroundcolor=col1!25
  ]{warning}{\color{col1!25!black}Note}

%% ----- Challange environment ----- %%

\mdfdefinestyle{theoremstyle}
{
  linewidth=2pt,
  frametitlerule=true,
  innertopmargin=\topskip,
}
\mdtheorem[
  style=theoremstyle,
  linecolor=col3!75,
  frametitlebackgroundcolor=col3!25
  ]{challange}{\color{col3!25!black}Challange}

%% ----- Important concept environment ----- %%

\mdfdefinestyle{theoremstyle}
{
  linewidth=2pt,
  frametitlerule=true,
  innertopmargin=\topskip,
}
\mdtheorem[
  style=theoremstyle,
  linecolor=red!50!black,
  frametitlebackgroundcolor=red
  ]{important}{IMPORTANT!}

%% ----- Example environment ----- %%

\mdfdefinestyle{theoremstyle}
{
  linewidth=2pt,
  frametitlerule=true,
  innertopmargin=\topskip,
}
\mdtheorem[
  style=theoremstyle,
  linecolor=col2!75,
  frametitlebackgroundcolor=col2!25
  ]{example}{\color{col2!25!black}Example}

%% ----- Misc. ----- %%

\newcommand{\true}{\textcolor{col2}{\textbf{true}}}
\newcommand{\false}{\textcolor{col1}{\textbf{false}}}
\newcommand{\mtrue}{\textcolor{col2}{\mathbf{true}}}
\newcommand{\mfalse}{\textcolor{col1}{\mathbf{false}}}
\newcommand{\mathcolorbox}[2]{\colorbox{#1}{$\displaystyle #2$}}
\newcommand{\thl}[1]{\mathcolorbox{col2!50}{#1}}
\newcommand{\fhl}[1]{\mathcolorbox{col1!50}{#1}}
\renewcommand{\emph}[1]{
  \colorbox{col4!30}{\textbf{#1}}\xspace
}

%% ----- VERY specific coloring macros ----- %%


\newcommand{\tmi}[2]{
  \tikzmarkin[set fill color=col#1!20, set border color=col#1]{#2}
}

\newcommand{\hli}[3]{
  \tmi{#1}{#2}#3\tikzmarkend{#2}
}

\newcommand{\xhl}[1][x]{
  \rcolor{col1}{#1}
}
\newcommand{\yhl}[1][y]{
  \rcolor{col2}{#1}
}
\newcommand{\zhl}[1][z]{
  \rcolor{col3}{#1}
}
\newcommand{\whl}[1][w]{
  \rcolor{col4}{#1}
}

\newcommandx{\mel}[3][1=a, 2=i, 3=j]{
  #1_{\xhl[#2]\yhl[#3]}
}

\newcommand{\matcl}[2]{
  \IfEqCase{#1}{
  {A}{\ \tmi{4}{#2}A\tikzmarkend{#2}\ }
  {B}{\ \tmi{5}{#2}B\tikzmarkend{#2}\ }
  {C}{\ \tmi{8}{#2}C\tikzmarkend{#2}\ }
  {MN}{\ \tmi{4}{#2}M\times N\tikzmarkend{#2}\ }
  {NK}{\ \tmi{5}{#2}N\times K\tikzmarkend{#2}\ }
  {MK}{\ \tmi{8}{#2}M\times K\tikzmarkend{#2}\ }
  }[\PackageError{tree}{Undefined option: #1}{}]
}

%% ----- Bra-Ket ----- %%
\newcommand{\bra}[1]{
	\langle| \left. #1 \right|
}

\newcommand{\ket}[1]{
	\left| #1 \right. \rangle
}
