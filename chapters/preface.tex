\chapter*{Preface}
\addcontentsline{toc}{chapter}{Preface}
This booklet contains the lecture notes for part I of a course given in the summer semester of 2020, at the Georg-August-Universit\"at in G\"ottingen, Germany. The course in question is called \emph{Mathematics and Computer Science (B.MES.108)}, and was given as part of the bachelor program \emph{Molecular Ecosystem Sciences} of the \emph{Faculty of Forest Sciences and Forest Ecology}, by the \emph{Department Ecoinformatics, Biometrics and Forest Growth}.

The course is intended on introducing the students, in their first year of studies, to basic mathematical tools that they would use in the course of their scientific career. \textbf{It is not intended to provide in-depth mathematical content}, but rather give a general overview of topics such as Linear algebra (given here), analysis of real functions and computer science. As such, the lectures do not dive deep into any particular topic, and are not to be used as an exhaustive reference on the matter (especially not from a mathematician's point of view, as they provide a more applicable approach rather than a pure theoretical one).

In this booklet, the reader will find notes on the different topics covered in the first part of the lectures. These notes are combined with practical examples, signified by a pale blue bounding box, such as follows:
\begin{example}
  This is a practical example.
\end{example}

In addition, two color-coded "boxes" can also be found: the first is a "warning"/"attention" box which stresses important notes and possible pitfalls to be aware of:
\begin{warning}
  This is an attention box.
\end{warning}

The second is a "challange" box, which challenges the reader to do some calculation by themselves, or to prove a lemma/theorem. These challenges are not stricly mandatory for understanding and/or successfuly studying the topic; it is however highly adviced to at least try them out.

\begin{challange}
  This is a challange box.
\end{challange}

In addition, some other boxes are used troughout the booklet more sparingly, such as a "definition" box, and a box that signifies extremely important concepts.

If you find any mistakes in the text (including typos and misspelling), please \href{pelegs@gmail.com}{contact me per email}, and let me know.

Lastly, unless otherwise mentioned, all graphics in this booklet were created by me, using the \href{https://github.com/pgf-tikz/pgf}{\TikZ\ package}, \href{http://ipe.otfried.org}{the Ipe extensible drawing editor} and \href{https://inkscape.org}{Inkscape}. The typesetting was done in \LaTeX. 
