\chapter{Linear Transformations}
\section{What is a Linear Transformation?}
A \emph{linear transformation}\index{Linear transformation} is a function $\func{T}{A}{B}$, that obeys the follows two criteria:
\begin{itemize}
  \item For each $x\in A$ and a scalar $\alpha$:
  \begin{equation*}
  T(\alpha x) = \alpha T(x).
  \end{equation*}
  \item For any $x,y\in A$:
  \begin{equation*}
  T(x+y) = T(x)+T(y).
  \end{equation*}
\end{itemize}

\begin{example}
  The function $\func{f}{\mathbb{R}}{\mathbb{R}},\ f(x)=3x$ is linear. Proof by the above criteria:
  \begin{itemize}
  \item For any scalar $\alpha\in\mathbb{R}$, 
  \begin{align*}
  f(\alpha x) &= 3(\alpha x)\\
  &= 3\alpha x\\
  &= \alpha \cdot 3x\\
  &= \alpha f(x).
  \end{align*}
  \item For any two numbers $x,y\in\mathbb{R}$
  \begin{align*}
  f(x+y) &= 3(x+y)\\
  &= 3x+3y\\
  &= f(x)+f(y).
  \end{align*}
  \end{itemize}
  Therefore, $f$ is linear.
\end{example}

\begin{example}
  Is the function $\func{g}{\mathbb{R}}{\mathbb{R}},\ g(x)=3x+5$ linear? Let's check:
  \begin{itemize}
  \item For a scalar $\alpha\in\mathbb{R}$: $g(\alpha x) = 3\alpha x + 5$, but $\alpha g(x)=\alpha(3x+5)=3\alpha x + 5\alpha$.\\
  For example, for $x=1, \alpha=2$:
  \begin{align*}
  g(\alpha x) &= g(2\cdot 1) = 6+5 = 11\\
  \alpha g(x) &= g(1) = 2(3+5) = 2\cdot8 = 16 \neq 11.
  \end{align*}
  \end{itemize}
  Since at least one of the criteria is not met (in this case the first criterion), $g$ is \textbf{not} linear.
\end{example}

\begin{warning}
  For a function to be non-linear, it is \textbf{enough} that any one of the two criteria isn't fulfilled.\\
  Contrary to that, for a function to be linear, \textbf{both} criteria must be fulfilled.
\end{warning}

\begin{challange}
  Check if $g$ fulfills the second criterion.
\end{challange}

\begin{example}
  Is the function $\func{h}{\mathbb{R}}{\mathbb{R}},\ h(x)=x^{2}$ linear? Let's check:
  \begin{itemize}
  \item For $x,y\in\mathbb{R}$: $h(x+y) = \left(x+y\right)^{2} = x^{2}+2xy+y^{2}$.\\
  On the other hand, $h(x)+h(y) = x^{2}+y^{2}$. For $x,y\neq0$ these are not equal.
  \end{itemize}
  Therefore, $h$ is \textbf{not} a linear transformation.
\end{example}

\begin{challange}
  Check if $h$ fulfills the first criterion.
\end{challange}

The two criteria for linearity can be combined together to a single criterion:
\begin{mathdef}
  A function $\func{T}{A}{B}$ is linear, if for any $x,y\in A$ and any scalars $\alpha, \beta$:
  \begin{equation*}
  T\left(\alpha x + \beta y\right) = \alpha T(x) + \beta T(y).
  \end{equation*}
\end{mathdef}

\section{Transforming Vectors}
Vectors can also be transformed, specifically by functions of the type $\func{T}{\Rs{n}}{\Rs{m}}$, with $n, m\in\mathbb{N}$. In this course we will mostly concentrate on transformations where the dimensions $n,m$ are each either $1,2$ or $3$ (i.e. transformations of 2- or 3-dimensional vectors, which result in 1-,2- or 3-dimensional vectors). These transformations are much easier to conceptualize (and infintely easier to draw).

\begin{example}
  A transformation $\func{T}{\Rs{2}}{\Rs{2}}$ is defined as:
  \begin{equation*}
  T(\vec{v}) = T(v_{x}, v_{y}) = \colvec{2}{-v_{x}}{3v_{y}},
  \end{equation*}
  i.e. it takes the vector $\vec{v}=\colvec{2}{v_{x}}{v_{y}}$, horizontally "flips" its $x$-component and scales its $y$-component by 3.

  Some examples:
  \begin{equation*}
  \colvec{2}{3}{1}\overset{T}{\longrightarrow}\colvec{2}{-3}{3},\quad\colvec{2}{-1}{2}\overset{T}{\longrightarrow}\colvec{2}{1}{6},\quad\colvec{2}{-7}{\frac{1}{3}}\overset{T}{\longrightarrow}\colvec{2}{7}{1},\quad\colvec{2}{0}{0}\overset{T}{\longrightarrow}\colvec{2}{0}{0}
  \end{equation*}
 
  The specific trasformation of the vector $\vec{v}=\colvec{2}{3}{1}$ looks graphically as follows:
  \begin{figure}[H]
  \centering
  \begin{tikzpicture}
  \Large

  \coordinate (o) at (0,0);
  \coordinate (u) at (3,1);
  \coordinate (Tu) at (-3,3);

  \drawaxes{-4}{-1}{4}{4}

  \draw[->, >=stealth, very thick, col1] (o) --(u) node [above right, yshift=-1mm] {$\vec{v}$};
  \draw[->, >=stealth, very thick, col2] (o) --(Tu) node [above] {$T(\vec{v})$};

  \draw[->, >=stealth, ultra thick, dashed, col4] (2,1) to [out=120, in=30] (-1.5,1.75);
  \node at (0.5, 2.5) {\color{col4}$T$};
  \end{tikzpicture}
  \end{figure}
\end{example}

\begin{warning}
  Remember that vectors are always drawn starting from the origin!
\end{warning}

\section{Transforming Spaces}
To visualize the way the entire space transforms by a transformation $T$, we can look at many points transforming together. Since each point represents a vector from the origin to that point, this will show us how different vectors are transformed. For this purpose, we can take the intersections of all the grid lines, then transforming them (here shown in blue), and draw the grid lines between them.

\begin{example}
  The transformation $T$ is defined by its action on a vector $\colvec{2}{x}{y}$ as follows:
  \begin{equation*}
  T\colvec{2}{x}{y} = \colvec{2}{0.35x-0.55y}{0.7x+0.65y}.
  \end{equation*}

  Graphically, this is how the transformation affects $\Rs{2}$ (the red dots on the left are lying on the integer grid points, and are transformed to the blue dots on the right):
  \begin{figure}[H]
  \centering
  \begin{tikzpicture}[scale=0.8]
  \Large

  \coordinate (shift) at (9, 0);
  \pgfmathsetmacro{\a}{0.35}
  \pgfmathsetmacro{\b}{-0.55}
  \pgfmathsetmacro{\c}{0.7}
  \pgfmathsetmacro{\d}{0.65}
  \coordinate (xhat) at (\a,\c);
  \coordinate (yhat) at (\b,\d);
  
  \drawaxes{-3}{-3}{3}{3}
  \drawaxes{-3}{-3}{3}{3}[9]

  \foreach \x in {-3,...,3}{
  \foreach \y in {-3,...,3}{
  % Original vector
  \coordinate (v) at (\x, \y);
  % Transformed vector
  \coordinate (vnew) at (\a*\x+\b*\y, \c*\x+\d*\y);

  % Draw new grid lines
  % (transformed v + transformed basis vectors)
  
  \draw[-, dotted] (vnew)++(shift) -- ++(xhat);
  \draw[-, dotted] (vnew)++(shift) -- ++(yhat);
   
  % Draw new points
  \filldraw[col1] (v) circle (0.1);
  \filldraw[col2] (vnew)++(shift) circle (0.1);
  }
  }

  \draw[->, >=stealth, very thick, dashed, col4] (2,3.5) to [out=45, in=135] (7,3.5);
  \node at (4.5,5) {\color{col4}$T$};

  \node at (0,-5) {Original space};
  \node at (9,-5) {Transformed space ($\Rs{2}$)};
  \end{tikzpicture}
  \end{figure}
\end{example}

The example above is a linear transformation. Quick proof:
\begin{itemize}
  \item $T$ is scalable:
  \begin{align*}
  T\left(\alpha\colvec{2}{x}{y}\right) &= T\colvec{2}{\alpha x}{\alpha y} = \colvec{2}{0.35\alpha x - 0.55\alpha y}{0.7\alpha x+0.65\alpha y} = \colvec{2}{\alpha\left(0.35x-0.55y\right)}{\alpha\left(0.7x+0.65y\right)}\\
  &= \alpha\colvec{2}{0.35x-0.55y}{0.7x+0.65y}=\alpha T\colvec{2}{x}{y}.
  \end{align*}
  \item $T$ is separable:
  \begin{align*}
  T\left( \colvec{2}{x}{y} + \colvec{2}{a}{b} \right) &= T\colvec{2}{x+a}{y+b} = \colvec{2}{0.35\left( x+a \right) - 0.55\left( y+b \right)}{0.7\left( x+a \right)+0.65\left( y+b \right)} = \colvec{2}{0.35x+0.35a-0.55y-0.55b}{0.7x+0.7a+0.65y+0.65b}\\
  &= \colvec{2}{0.35x-0.55y}{0.7x+0.65y} + \colvec{2}{0.35a-0.55b}{0.7a+0.65b} = T\colvec{2}{x}{y} + T\colvec{2}{a}{b}.
  \end{align*}
\end{itemize}

\section{Characteristics of Linear Transformations}
The graphical representation of the effect of $T$ on $\Rs{2}$ above reveals several important characteristics of linear transformations:
\begin{itemize}
  \item Linear transformations always preserve the zero vector, i.e. they always map $\vec{0}$ to $\vec{0}$ (in the case of $\Rs{2}$: $\colvec{2}{0}{0}\xrightarrow[]{T}\colvec{2}{0}{0}$).
  \item Parallel lines remain parallel after the transformation is applied.
  \item While areas can scale under the transformation, the ratio between areas is preserved.
\end{itemize}

\begin{challange}
  Show that all three characteristics above can be derived directly from the definitions of a linear transformation.
\end{challange}

\section{Basic Types of Linear Trasformations}
In $\Rs{2}$ there are several basic linear transformation from which all other linear transformations are constructed (via composition).

\subsection{Rotations Around the Origin}
Rotating space around its origin counter clockwise by an angle $\theta$. The transformation is usually denoted as $R_{\theta}$.
\begin{figure}[H]
  \centering
  \begin{tikzpicture}[scale=0.8]
  \Large

  \coordinate (shift) at (9, 0);
  \pgfmathsetmacro{\a}{0.886}
  \pgfmathsetmacro{\b}{-0.5}
  \pgfmathsetmacro{\c}{0.5}
  \pgfmathsetmacro{\d}{0.886}
  \coordinate (xhat) at (\a,\c);
  \coordinate (yhat) at (\b,\d);
  
  \drawaxes{-3}{-3}{3}{3}
  \drawaxes{-3}{-3}{3}{3}[9]

  \foreach \x in {-3,...,3}{
  \foreach \y in {-3,...,3}{
  % Original vector
  \coordinate (v) at (\x, \y);
  % Transformed vector
  \coordinate (vnew) at (\a*\x+\b*\y, \c*\x+\d*\y);

  % Draw new grid lines
  % (transformed v + transformed basis vectors)
  
  \draw[-, dotted] (vnew)++(shift) -- ++(xhat);
  \draw[-, dotted] (vnew)++(shift) -- ++(yhat);
   
  % Draw new points
  \filldraw[col1] (v) circle (0.1);
  \filldraw[col2] (vnew)++(shift) circle (0.1);
  }
  }
  \draw[->, >=stealth, very thick, dashed, col4] (2,3.5) to [out=45, in=135] (7,3.5);
  \node at (4.5,5) {\color{col4}$T=R_{\theta}$};
  \end{tikzpicture}
\end{figure}

\subsection{Scaling}
The scaling of space by a scalar is a linear translation. In addition, scaling space by a scalar in either the $x$ or $y$ direction is also a linear transformation. We can denote a scaling transformation by $S_{\alpha}$, where $\alpha$ is the scalar. Similarily, scaling in the $x$ direction would be $S^{x}_{\alpha}$, and in the $y$ direction $S^{y}_{\alpha}$.

\begin{warning}
In the following graphical representations all scalings are done by scalars smaller than 1 for graphical reasons, but scalings can of course be done for any scalar $\alpha\in\mathbb{R}$).
\end{warning}

\begin{figure}[H]
  \centering
  \begin{tikzpicture}[scale=0.8]
  \Large

  \coordinate (shift) at (9, 0);
  \pgfmathsetmacro{\a}{0.75}
  \pgfmathsetmacro{\b}{0}
  \pgfmathsetmacro{\c}{0}
  \pgfmathsetmacro{\d}{0.75}
  \coordinate (xhat) at (\a,\c);
  \coordinate (yhat) at (\b,\d);
  
  \drawaxes{-3}{-3}{3}{3}
  \drawaxes{-3}{-3}{3}{3}[9]

  \foreach \x in {-3,...,3}{
  \foreach \y in {-3,...,3}{
  % Original vector
  \coordinate (v) at (\x, \y);
  % Transformed vector
  \coordinate (vnew) at (\a*\x+\b*\y, \c*\x+\d*\y);

  % Draw new grid lines
  % (transformed v + transformed basis vectors)
  
  \draw[-, dotted] (vnew)++(shift) -- ++(xhat);
  \draw[-, dotted] (vnew)++(shift) -- ++(yhat);
   
  % Draw new points
  \filldraw[col1] (v) circle (0.1);
  \filldraw[col2] (vnew)++(shift) circle (0.1);
  }
  }
  \draw[->, >=stealth, very thick, dashed, col4] (2,3.5) to [out=45, in=135] (7,3.5);
  \node at (4.5,5) {\color{col4}$T=S_{\alpha}$};
  \end{tikzpicture}
\end{figure}

Scaling in one direction only:
\begin{figure}[H]
  \centering
  \begin{tikzpicture}[scale=0.8]
  \Large

  \coordinate (shift) at (9, 0);
  \pgfmathsetmacro{\a}{0.75}
  \pgfmathsetmacro{\b}{0}
  \pgfmathsetmacro{\c}{0}
  \pgfmathsetmacro{\d}{1}
  \coordinate (xhat) at (\a,\c);
  \coordinate (yhat) at (\b,\d);
  
  \drawaxes{-3}{-3}{3}{3}
  \drawaxes{-3}{-3}{3}{3}[9]

  \foreach \x in {-3,...,3}{
  \foreach \y in {-3,...,3}{
  % Original vector
  \coordinate (v) at (\x, \y);
  % Transformed vector
  \coordinate (vnew) at (\a*\x+\b*\y, \c*\x+\d*\y);

  % Draw new grid lines
  % (transformed v + transformed basis vectors)
  
  \draw[-, dotted] (vnew)++(shift) -- ++(xhat);
  \draw[-, dotted] (vnew)++(shift) -- ++(yhat);
   
  % Draw new points
  \filldraw[col1] (v) circle (0.1);
  \filldraw[col2] (vnew)++(shift) circle (0.1);
  }
  }
  \draw[->, >=stealth, very thick, dashed, col4] (2,3.5) to [out=45, in=135] (7,3.5);
  \node at (4.5,5) {\color{col4}$T=S^{x}_{\alpha}$};
  \end{tikzpicture}
\end{figure}
\begin{figure}[H]
  \centering
  \begin{tikzpicture}[scale=0.8]
  \Large

  \coordinate (shift) at (9, 0);
  \pgfmathsetmacro{\a}{1}
  \pgfmathsetmacro{\b}{0}
  \pgfmathsetmacro{\c}{0}
  \pgfmathsetmacro{\d}{0.5}
  \coordinate (xhat) at (\a,\c);
  \coordinate (yhat) at (\b,\d);
  
  \drawaxes{-3}{-3}{3}{3}
  \drawaxes{-3}{-3}{3}{3}[9]

  \foreach \x in {-3,...,3}{
  \foreach \y in {-3,...,3}{
  % Original vector
  \coordinate (v) at (\x, \y);
  % Transformed vector
  \coordinate (vnew) at (\a*\x+\b*\y, \c*\x+\d*\y);

  % Draw new grid lines
  % (transformed v + transformed basis vectors)
  
  \draw[-, dotted] (vnew)++(shift) -- ++(xhat);
  \draw[-, dotted] (vnew)++(shift) -- ++(yhat);
   
  % Draw new points
  \filldraw[col1] (v) circle (0.1);
  \filldraw[col2] (vnew)++(shift) circle (0.1);
  }
  }
  \draw[->, >=stealth, very thick, dashed, col4] (2,3.5) to [out=45, in=135] (7,3.5);
  \node at (4.5,5) {\color{col4}$T=S^{y}_{\alpha}$};
  \end{tikzpicture}
\end{figure}
%  \item Scaling - either of each component separately, or of both together:
%  \begin{enumerate}
% \item Scaling both components together:
% \begin{figure}[H]
% \centering
% \begin{tikzpicture}
% \coordinate (shift) at (9, 0);
% \pgfmathsetmacro{\a}{0.5}
% \pgfmathsetmacro{\b}{0}
% \pgfmathsetmacro{\c}{0}
% \pgfmathsetmacro{\d}{0.5}
% \coordinate (xhat) at (\a,\c);
% \coordinate (yhat) at (\b,\d);
% 
% \drawaxes{-3}{-3}{3}{3}
% \drawaxes{-3}{-3}{3}{3}[9]
%
% \foreach \x in {-3,...,3}{
% \foreach \y in {-3,...,3}{
% % Original vector
% \coordinate (v) at (\x, \y);
% % Transformed vector
% \coordinate (vnew) at (\a*\x+\b*\y, \c*\x+\d*\y);
%
% % Draw new grid lines
% % (transformed v + transformed basis vectors)
% 
% \draw[-, dotted] (vnew)++(shift) -- ++(xhat);
% \draw[-, dotted] (vnew)++(shift) -- ++(yhat);
%  
% % Draw new points
% \filldraw[col1] (v) circle (0.1);
% \filldraw[col2] (vnew)++(shift) circle (0.1);
% }
% }
% \draw[->, >=stealth, very thick, dashed, col4] (2,3.5) to [out=45, in=135] (7,3.5);
% \node at (4.5,5) {\color{col4}$T$};
% \end{tikzpicture}
% \end{figure}
%
% \item Scaling only in the $x$-direction:
% \begin{figure}[H]
% \centering
% \begin{tikzpicture}
% \coordinate (shift) at (9, 0);
% \pgfmathsetmacro{\a}{0.5}
% \pgfmathsetmacro{\b}{0}
% \pgfmathsetmacro{\c}{0}
% \pgfmathsetmacro{\d}{1}
% \coordinate (xhat) at (\a,\c);
% \coordinate (yhat) at (\b,\d);
% 
% \drawaxes{-3}{-3}{3}{3}
% \drawaxes{-3}{-3}{3}{3}[9]
%
% \foreach \x in {-3,...,3}{
% \foreach \y in {-3,...,3}{
% % Original vector
% \coordinate (v) at (\x, \y);
% % Transformed vector
% \coordinate (vnew) at (\a*\x+\b*\y, \c*\x+\d*\y);
%
% % Draw new grid lines
% % (transformed v + transformed baise vectors)
% 
% \draw[-, dotted] (vnew)++(shift) -- ++(xhat);
% \draw[-, dotted] (vnew)++(shift) -- ++(yhat);
%  
% % Draw new points
% \filldraw[col1] (v) circle (0.1);
% \filldraw[col2] (vnew)++(shift) circle (0.1);
% }
% }
% \draw[->, >=stealth, very thick, dashed, col4] (2,3.5) to [out=45, in=135] (7,3.5);
% \node at (4.5,5) {\color{col4}$T$};
% \end{tikzpicture}
% \end{figure}
%
% \item Scaling only in the $y$-direction:
% \begin{figure}[H]
% \centering
% \begin{tikzpicture}
% \coordinate (shift) at (9, 0);
% \pgfmathsetmacro{\a}{1}
% \pgfmathsetmacro{\b}{0}
% \pgfmathsetmacro{\c}{0}
% \pgfmathsetmacro{\d}{0.5}
% \coordinate (xhat) at (\a,\c);
% \coordinate (yhat) at (\b,\d);
% 
% \drawaxes{-3}{-3}{3}{3}
% \drawaxes{-3}{-3}{3}{3}[9]
%
% \foreach \x in {-3,...,3}{
% \foreach \y in {-3,...,3}{
% % Original vector
% \coordinate (v) at (\x, \y);
% % Transformed vector
% \coordinate (vnew) at (\a*\x+\b*\y, \c*\x+\d*\y);
%
% % Draw new grid lines
% % (transformed v + transformed baise vectors)
% 
% \draw[-, dotted] (vnew)++(shift) -- ++(xhat);
% \draw[-, dotted] (vnew)++(shift) -- ++(yhat);
%  
% % Draw new points
% \filldraw[col1] (v) circle (0.1);
% \filldraw[col2] (vnew)++(shift) circle (0.1);
% }
% }
% \draw[->, >=stealth, very thick, dashed, col4] (2,3.5) to [out=45, in=135] (7,3.5);
% \node at (4.5,5) {\color{col4}$T$};
% \end{tikzpicture}
% \end{figure}
%  \end{enumerate}
%
%  \item Reflection of the plane by a line going through the origin (example line drawn here as a black dashed line):
%  \begin{figure}[H]
% \centering
% \begin{tikzpicture}
% \tikzstyle{stern} = [star, star points=5, star point ratio=0.5, draw=black]
%
% \coordinate (shift) at (9, 0);
% \pgfmathsetmacro{\a}{0}
% \pgfmathsetmacro{\b}{1}
% \pgfmathsetmacro{\c}{1}
% \pgfmathsetmacro{\d}{0}
% \coordinate (xhat) at (\a,\c);
% \coordinate (yhat) at (\b,\d);
% 
% \drawaxes{-3}{-3}{3}{3}
% \drawaxes{-3}{-3}{3}{3}[9]
%
% % Draw reflection line
% \draw[-, dashed, thick] (-3,-3) to (3,3);
% \draw[-, dashed, thick] ($(-3,-3)+(shift)$) to ($(3,3)+(shift)$);
%
% \foreach \x in {-3,...,3}{
% \foreach \y in {-3,...,3}{
% % Original vector
% \coordinate (v) at (\x, \y);
% % Transformed vector
% \coordinate (vnew) at (\a*\x+\b*\y, \c*\x+\d*\y);
%
% % Draw new grid lines
% % (transformed v + transformed baise vectors)
% \draw[-, dotted] (vnew)++(shift) -- ++(xhat);
% \draw[-, dotted] (vnew)++(shift) -- ++(yhat);
%  
% % Draw new points
% \filldraw[col1] (v) circle (0.1);
% \filldraw[col2] (vnew)++(shift) circle (0.1);
%
% % Draw some colored points on the original space
% \node[stern, fill=gray] at (-2,2) {};
% \node[stern, fill=col3] at (2,1) {};
% \node[stern, fill=col7] at (1,1) {};
% \node[stern, fill=col2] at (1,-1) {};
%
% % Draw some colored points on the transformed space
% \node[stern, fill=gray] at ($(2,-2)+(shift)$){};
% \node[stern, fill=col3] at ($(1,2) +(shift)$){};
% \node[stern, fill=col7] at ($(1,1) +(shift)$){};
% \node[stern, fill=col2] at ($(-1,1) +(shift)$){};
% }
% }
% \draw[->, >=stealth, very thick, dashed, col4] (2,3.5) to [out=45, in=135] (7,3.5);
% \node at (4.5,5) {\color{col4}$T$};
% \end{tikzpicture}
%  \end{figure}
%
%  \item Shear (Skew) transformation:
%  \begin{figure}[H]
% \centering
% \begin{tikzpicture}
% \coordinate (shift) at (9, 0);
% \pgfmathsetmacro{\a}{1}
% \pgfmathsetmacro{\b}{.5}
% \pgfmathsetmacro{\c}{0}
% \pgfmathsetmacro{\d}{1}
% \coordinate (xhat) at (\a,\c);
% \coordinate (yhat) at (\b,\d);
% 
% \drawaxes{-3}{-3}{3}{3}
% \drawaxes{-3}{-3}{3}{3}[9]
%
% \foreach \x in {-3,...,3}{
% \foreach \y in {-3,...,3}{
% % Original vector
% \coordinate (v) at (\x, \y);
% % Transformed vector
% \coordinate (vnew) at (\a*\x+\b*\y, \c*\x+\d*\y);
%
% % Draw new grid lines
% % (transformed v + transformed baise vectors)
% \draw[-, dotted] (vnew)++(shift) -- ++(xhat);
% \draw[-, dotted] (vnew)++(shift) -- ++(yhat);
%  
% % Draw new points
% \filldraw[col1] (v) circle (0.1);
% \filldraw[col2] (vnew)++(shift) circle (0.1);
% }
% }
% \draw[->, >=stealth, very thick, dashed, col4] (2,3.5) to [out=45, in=135] (7,3.5);
% \node at (4.5,5) {\color{col4}$T$};
% \end{tikzpicture}
%  \end{figure}
%\end{itemize}

TODO: ADD MORE TRANSFORMATIONS

\begin{warning}
  Addition of some constant non-zero vector, known as a \emph{translation}\index{Translation (transformation)}, is \textbf{not} a linear transformation. These transformations are a type of \emph{affine}\index{Affine transformation} transformations which are not discussed in this course, and can be represented as a linear transformation in higher dimensions (as we will see later in the course).
\end{warning}

