\sectionpic{Linear Algebra, Calculus and (a bit of) Quatum Physics}{../figures/presentation_chapters/quantum.pdf}

\begin{frame}
	\frametitle{Classical Mechanics}
	\only<1>{
		In \emph{classical mechanics} we charecterize a point-like particle as having a well-defined \emph{position} represented as an $\Rs{3}$ vector:
	}
	\only<2>{
		A particle also has a \emph{velocity} (change of position over time), which is also represented by an $\Rs{3}$ vector:
	}

	\vspace{1cm}
	\centering
	\begin{tikzpicture}
		\begin{axis}[
				Axis Style,
				width=5cm, height=5cm,
				axis line style={->, -stealth, thick},
				xmin=0, xmax=2,
				ymin=0, ymax=2,
				ticks=none,
			]
				\draw[vector, dashed, thick, black!50] (axis cs: 0,0) to (axis cs: 1,1.5) node [xshift=3mm] {$\vec{r}$};
			\only<2>{
				\draw[vector, thick, black!50] (axis cs: 1,1.5) to (axis cs: 1.3,1) node [xshift=3mm] {$\vec{v}$};
			}
			\node[circle, fill=col1, inner sep=0pt, minimum size=3pt] at (axis cs: 1,1.5) {};
		\end{axis}
	\end{tikzpicture}
\end{frame}
