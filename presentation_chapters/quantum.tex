\sectionpic{Linear Algebra, Calculus and (a bit of) Quatum Physics}{../figures/presentation_chapters/quantum.pdf}

\begin{frame}
	\frametitle{Classical Mechanics}
	\only<1>{
		In \emph{classical mechanics} we charecterize a point-like particle as having a well-defined \emph{position} represented as an $\Rs{3}$ vector:
	}
	\only<2>{
		A particle also has a \emph{velocity} (change of position over time), which is also represented by an $\Rs{3}$ vector:
	}

	\vspace{1cm}
	\centering
	\begin{tikzpicture}
		\begin{axis}[
				Axis Style,
				width=5cm, height=5cm,
				axis line style={->, -stealth, thick},
				xmin=0, xmax=2,
				ymin=0, ymax=2,
				ticks=none,
			]
				\draw[vector, dashed, thick, black!50] (axis cs: 0,0) to (axis cs: 1,1.5) node [xshift=3mm] {$\vec{r}$};
			\only<2>{
				\draw[vector, thick, black!50] (axis cs: 1,1.5) to (axis cs: 1.3,1) node [xshift=3mm] {$\vec{v}$};
			}
			\node[circle, fill=col1, inner sep=0pt, minimum size=3pt] at (axis cs: 1,1.5) {};
		\end{axis}
	\end{tikzpicture}
\end{frame}

\begin{frame}
	\frametitle{Classical Mechanics}
	The position and velocity of a particle are time-dependent, i.e. they are functions of the time $t$:
	\begin{align*}
		\vec{r}(t) &= \colvec{3}{x(t)}{y(t)}{z(t)},\\
		\vec{v}(t) &= \colvec{3}{v_{x}(t)}{v_{y}(t)}{v_{z}(t)}.
	\end{align*}
\end{frame}

\begin{frame}
	\frametitle{Classical Mechanics}
	Each component of $\vec{v}$ is the time derivative of the respective component of $\vec{r}$:
	\begin{align*}
		v_{x}(t) &= \od{x(t)}{t},\\
		v_{y}(t) &= \od{y(t)}{t},\\
		v_{z}(t) &= \od{z(t)}{t}.
	\end{align*}
\end{frame}

\begin{frame}
	\frametitle{Classical Mechanics}
	For example, if we plot $x(t)$ over $t$ (i.e. the $x$-position of a particle over time), the derivative at some time $t_{0}$ is the $x$-component of the velocity of the partile at that time:

	\centering
	\begin{tikzpicture}[node distance=1mm]
		\begin{axis}[
				Axis Style,
				xlabel={$t$},
				ylabel={$x(t)$},
				width=7.5cm, height=7.5cm,
				axis line style={->, -stealth, thick},
				xmin=-2, xmax=2,
				ymin=-2, ymax=2,
				ticks=none,
			]
			\addplot[funcdraw, col1, ultra thick] {traj(x)};

			\pgfmathsetmacro{\xa}{-0.5}
			\node[circle, inner sep=1pt, minimum size=0.1pt, fill=black] (xval) at (axis cs:\xa,0) {};
			\node[below=of xval] {$t_{0}$};
			\node[circle, inner sep=1pt, minimum size=0.1pt, fill=black] (yval) at (axis cs:\xa,{traj(\xa)}) {};
			\node[above=of yval] {$x(t_{0})$};
			\path[draw=black, dashed] (xval) to (yval);
			
			\pgfmathsetmacro{\dx}{0.5}
			\pgfmathsetmacro{\dy}{dtraj(\xa)*\dx}
			\path[draw=col2, dashed] (axis cs:\xa,{traj(\xa)}) to (axis cs:{\xa+\dx},{traj(\xa)+\dy});
		\end{axis}
	\end{tikzpicture}
	MORE STUFF
\end{frame}

\begin{frame}
	\frametitle{Quantum Mechanics}
	\emph{Quantum mechanics}, on the other hand, handles physical systems differently: a particle has no specific place nor velocity, but instead it has a function $\func{\Psi}{\Rs{3}}{\mathbb{C}}$ which completely encodes all the information about the particle at any given time.

	This kind of function is called a \emph{wave function}.
\end{frame}

\begin{frame}
	\frametitle{Quantum Mechanics}
	The wave function of a particle by itself has no physical meaning.

	However, the product $\Psi\Psi^{*}=|\Psi|^{2}$ (i.e. the product of the function with its complex-conjugate) \textbf{does} have a physical interpretation: at any given point $\vec{r}$, the value $|\Psi(\vec{r})|^{2}$ is the \emph{probability density} to find the 
\end{frame}
