\sectionpic{Integrals}{../figures/presentation_chapters/integrals.pdf}

\pgfmathsetmacro{\mu}{1}
\pgfmathsetmacro{\sig}{2}
\pgfmathsetmacro{\muB}{2}
\pgfmathsetmacro{\sigB}{0.3}
\tikzset{
	operator/.style={circle, inner sep=0pt, minimum size=1cm, text opacity=1, fill opacity=0.5},
	arrow/.style={->, >=stealth},
				}
\pgfkeys{/pgfplots/Axis Style/.style={
    axis x line=middle,
    axis y line=left,
		every axis x label/.style={at={(current axis.right of origin)}, anchor=west},
		every axis y label/.style={at={(axis description cs:0,1)}, anchor=south},
    samples=200,
		xmin=-5, xmax=5,
		ymin=-0.5, ymax=2,
		domain=-5:5,
		axis line style={thick},
		label style={font=\large},
		tick label style={font=\tiny},
		declare function={gauss(\x,\y,\z) = exp(-(\x-\y)^2/(2*\z^2));},
		declare function={func(\x) = gauss(\x, \mu, \sig);},
		declare function={gaussderiv(\x,\y,\z) = -(\x-\y)*gauss(\x,\y,\z)/\z^2;},
		declare function={funcderiv(\x) = gaussderiv(\x, \mu, \sig);},
}}

\begin{frame}
	\frametitle{Reminder: The Derivative}
	\centering
	
	\begin{presentation_definition}
		The \emph{derivative} of a function $f$ at the point $x_{0}$, denoted $f'(x_{0})$, is the \textbf{slope} of the tangent line to the function at $x_{0}$.
	\end{presentation_definition}
	
	\begin{tikzpicture}
		\begin{axis}[
				Axis Style,
				ticks=none,
			]
			\pgfmathsetmacro{\xa}{0.3}
			\pgfmathsetmacro{\dx}{0.1}
			\pgfmathsetmacro{\Dx}{2.0}
			\pgfmathsetmacro{\dxdy}{funcderiv(\xa)}
			\addplot[name path=f, mark=none, very thick, col2] {func(x)};
			\addplot[black, mark=*, only marks] coordinates {(\xa,{func(\xa)})};
			\path[draw=col1, thick, dashed] (axis cs:{\xa-\Dx},{func(\xa)-\dxdy*\Dx}) -- (axis cs:{\xa+\Dx},{func(\xa)+\dxdy*\Dx});
			\path[draw=black!30, dashed] (axis cs:\xa,{func(\xa)}) -- (axis cs:\xa,0) node [below] {$x_{0}$};
		\end{axis}
	\end{tikzpicture}
\end{frame}

\begin{frame}
	\frametitle{Reminder: The Derivative}
	\centering
	
	We can find the derivative by taking closer and closer points to $x_{0}$.
	
	\begin{tikzpicture}
		\begin{axis}[
				Axis Style,
			]
			\pgfmathsetmacro{\xa}{0.3}
			\pgfmathsetmacro{\dx}{0.1}
			\pgfmathsetmacro{\Dx}{2.0}
			\pgfmathsetmacro{\dxdy}{funcderiv(\xa)}
			\addplot[name path=f, mark=none, very thick, col2] {func(x)};
			\addplot[black, mark=*, only marks] coordinates {(\xa,{func(\xa)})};
			\path[draw=col1, thick, dashed] (axis cs:{\xa-\Dx},{func(\xa)-\dxdy*\Dx}) -- (axis cs:{\xa+\Dx},{func(\xa)+\dxdy*\Dx});
			\path[draw=black!30, dashed] (axis cs:\xa,{func(\xa)}) -- (axis cs:\xa,0) node [below] {$x_{0}$};
		\end{axis}
	\end{tikzpicture}
\end{frame}

\begin{frame}
	\frametitle{Reminder: The Derivative}
	Thus, the derivative is the limit where $\Delta x$ goes to $0$:
	\begin{presentation_definition}
		\begin{equation*}
			f'(x_{0}) = \lim\limits_{\Delta x\rightarrow0} \frac{f(x_{0}+\Delta x) - f(x_{0})}{\cancel{x_{0}} + \Delta x - \cancel{x_{0}}} = \lim\limits_{\Delta x\rightarrow0} \frac{\Delta y}{\Delta x} = \od{y}{x}.
		\end{equation*}
	\end{presentation_definition}
	\begin{presentation_note}
		The notation $\od{y}{x}$ \textbf{is not a fraction}. It only signifies that the derivative is the limit of $\frac{\Delta y}{\Delta x}$ when $\Delta x\rightarrow0$.
	\end{presentation_note}
\end{frame}

\begin{frame}
	\frametitle{Reminder: The Derivative}
	We can view the derivative as an \emph{operator} acting on a function:

	\vspace{5mm}
	\centering
	\begin{tikzpicture}
			\node[operator, draw=col3, fill=col3] (deriv) {$\od{}{x}$};
			\node[left=of deriv] (f1) {$f(x)$};
			\node[right=of deriv] (f1') {$f'(x)$};
			\draw[arrow] (f1) -- (deriv);
			\draw[arrow] (deriv) -- (f1');
	\end{tikzpicture}
	\flushleft

	\vspace{5mm}
	Recall that the following two statements are true:
	\begin{equation*}
		\begin{cases}
			\left( \alpha f \right)' &= \alpha f',\\
			\left( f+g \right)' &= f' + g'.\\
		\end{cases}
	\end{equation*}

	Similar to \emph{linear transformations} acting on vectors, this means that the derivative operator is \textbf{linear}.
\end{frame}

\begin{frame}
	\frametitle{Reminder: The Derivative}
	By viewing the derivative as an operator acting on functions, the notion of \emph{higher order derivatives} becomes pretty clear:
	\begin{presentation_definition}
		An $n$-th order derivative (where $n\in\mathbb{N}$) of a function $f$ is the result of applying the derivative operator $n$ consevutive times on $f$.
	\end{presentation_definition}

	We denote the $n$-th order derivative of $f$ as either
	\begin{equation*}
		f^{(n)},
	\end{equation*}
	or
	\begin{equation*}
		\od[n]{f}{x}.
	\end{equation*}
\end{frame}

\begin{frame}
	\frametitle{The Antiderivative}
	Similar to linear transformations (with a non-zero determinant), one might try to find an inverse operator to derivative. We call such an operator the \emph{Antiderivative}, or the \emph{Non-definite integral}.

	We can think of it a derivative of order $-1$, i.e. $\od[-1]{}{x}$:

	\vspace{5mm}
	\centering
	\begin{tikzpicture}
		\node[operator, draw=col2, fill=col2, below=of deriv] (antideriv) {$\od[-1]{}{x}$};
		\node[left=of antideriv] (f2) {$f'(x)$};
		\node[right=of antideriv] (f2') {$f(x)+C$};
		\draw[arrow] (f2) -- (antideriv);
		\draw[arrow] (antideriv) -- (f2');
		\node[below=of f2', text=col1] (why+c) {Why?};
		\draw[arrow, col1] (why+c) -- ($(f2')+(0.3,-0.3)$);
	\end{tikzpicture}
\end{frame}

\begin{frame}
	\frametitle{The Antiderivative}
	Recall that constants are derived to $0$, and thus don't affect the derivative:
	\begin{equation*}
		\left( f(x) + C \right)' = f'(x) + \cancel{C'} = f'(x).
	\end{equation*}

	Therefore the antiderivative of a function is not a function itself, but a \textbf{family} of functions, differing from each other by a constant:
	\begin{equation*}
		\od[-1]{}{x}f = \left\{ f(x)+C \mid C\in\mathbb{R} \right\}.
	\end{equation*}
\end{frame}

\begin{frame}
	\frametitle{The Antiderivative}
	A common way to denote the antiderivative of a function $f$ by the variable x is
	\begin{equation*}
		\int f(x) \dif x,
	\end{equation*}
	which is called the \emph{indefinite integral}.

	The reason for "multiplying" the function by $\dif x$ will be made clear later.
\end{frame}

\pgfmathsetmacro{\a}{-0.5}
\pgfmathsetmacro{\b}{0.5}
\pgfmathsetmacro{\mu}{0}
\pgfmathsetmacro{\sig}{0.4}
\begin{frame}
	\frametitle{The Definite Integral}
	What is the area under the curve of a function, between two points $x_{1}=a$ and $x_{2}=b$?

	\centering
	\begin{tikzpicture}
		\begin{axis}[
				Axis Style,
				xmin=-1, xmax=1,
				ymin=-0.3, ymax=1.1,
				domain=-1:1,
				xtick={\a, \b},
				xticklabels={$a$, $b$},
				ytick=\empty,
				tick label style={font=\large},
			]
			\path[name path=axis] (axis cs:-1,0) -- (axis cs:1,0);
			\addplot[name path=f, mark=none, ultra thick] {func(x)};
			\addplot[fill=col4, fill opacity=0.5] fill between [of=f and axis, soft clip={domain=\a:\b}];
			\draw[thick, dashed] (axis cs:\a,0) -- (axis cs:\a,{func(\a)});
			\draw[thick, dashed] (axis cs:\b,0) -- (axis cs:\b,{func(\b)});
		\end{axis}
	\end{tikzpicture}
\end{frame}

\begin{frame}[fragile]
	\frametitle{The Definite Integral}
	We can start by approximating the area with a shape of a known area: a rectangle.

	\begin{center}%
	\begin{tikzpicture}
		\begin{axis}[
				Axis Style,
				width=10cm, height=5cm,
				xmin=-1, xmax=1,
				ymin=-0.3, ymax=1.1,
				domain=-1:1,
				xtick={\a, \b},
				xticklabels={$a$, $b$},
				ytick=\empty,
				tick label style={font=\large},
			]
			\path[name path=axis] (axis cs:-1,0) -- (axis cs:1,0);
			\addplot[name path=f, mark=none, ultra thick] {func(x)};
			\draw[thick, dashed] (axis cs:\a,0) -- (axis cs:\a,{func(\a)});
			\draw[thick, dashed] (axis cs:\b,0) -- (axis cs:\b,{func(\b)});
			\pgfmathsetmacro{\N}{1}
			\pgfmathsetmacro{\dx}{(\b-\a)/\N}
			\pgfplotsinvokeforeach{\a,\a+\dx,...,\b-\dx}{
				\draw[thick, fill=col4!50] (axis cs:#1,0) rectangle (axis cs:#1+\dx,{min(func(#1), func(#1+\dx)}) node [midway] {Area: $f(a)\cdot(b-a)$};
			}
			\node (f_a) at (axis cs:\a-0.1, {func(\a)+0.1}) {$f(a)$};
		\end{axis}
	\end{tikzpicture}
	\end{center}
\end{frame}

\begin{frame}[fragile]
	\frametitle{The Definite Integral}
	\only<1>{
		Obviously, this is not a great approximation, so we can divide the interval $\left[ a,b \right]$ in three, and use the reulting rectangles (each with a base length $\Delta x=\frac{(b-a)}{3}$) to approximate the area:
	}
	\only<2>{	
		Although each of the rectangles have a base $\Delta x=\frac{(b-a)}{3}$, their areas are dependent on their position: their height is either the value of $f$ at their left side, or value of $f$ at their right side.
	}
	\only<3>{	
		This will be determined by choosing the heights as either the minimum or maximum of the left and right values of $f$. For now we will arbitrarily choose using the minimum values.
	}
	\only<4>{	
		Writing $x_{1}=a,\ x_{2}=a+\Delta x,\ x_{3}=a+2\Delta x,\ x_{4}=b$, the total area approximated by the rectangles is:
		\begin{align*}
			S_{\text{approx}} &= \min(f(x_{1}), f(x_{2}))(b-a) + \min(f(x_{2}), f(x_{3}))(b-a) + \min(f(x_{3}), f(x_{4}))(b-a)\\
			&= \left[\min(f(x_{1}), f(x_{2})) + \min(f(x_{2}), f(x_{3})) + \min(f(x_{3}), f(x_{4}))\right](b-a)\\
			&= \sum\limits_{i=1}^{3}\min\left( f\left( x_{i} \right), f\left( x_{i+1} \right) \right)\Delta x.
		\end{align*}
	}

	\only<5-9>{
		Of course, we can refine the approximation by increasing the number of rectangles (which is equivalent to reducing $\Delta x$, since $\Delta x=\frac{b-a}{N}$):
	}

	\begin{center}%
	\begin{tikzpicture}[scale=0.75]
		\begin{axis}[
				Axis Style,
				width=10cm, height=7.5cm,
				xmin=-1, xmax=1,
				ymin=-0.3, ymax=1.1,
				domain=-1:1,
				xtick={\a, \b},
				xticklabels={$a$, $b$},
				ytick=\empty,
				tick label style={font=\large},
			]
			\path[name path=axis] (axis cs:-1,0) -- (axis cs:1,0);
			\addplot[name path=f, mark=none, ultra thick] {func(x)};
			\draw[thick, dashed] (axis cs:\a,0) -- (axis cs:\a,{func(\a)});
			\draw[thick, dashed] (axis cs:\b,0) -- (axis cs:\b,{func(\b)});
			\only<1-4>{\pgfmathsetmacro{\N}{3}}
			\only<5>{\pgfmathsetmacro{\N}{5}}
			\only<6>{\pgfmathsetmacro{\N}{10}}
			\only<7>{\pgfmathsetmacro{\N}{15}}
			\only<8>{\pgfmathsetmacro{\N}{22}}
			\only<9>{\pgfmathsetmacro{\N}{30}}
			\pgfmathsetmacro{\dx}{(\b-\a)/\N}
			\pgfplotsinvokeforeach{\a,\a+\dx,...,\b-\dx}{
				\draw[thick, fill=col4!50] (axis cs:#1,0) rectangle (axis cs:#1+\dx,{min(func(#1), func(#1+\dx)});
			}
			\node at (axis cs:\a-0.2,{4*func(\a-0.2)}) {\huge$N=\N$};
			\node (f_a) at (axis cs:\a-0.1, {func(\a)}) {$f(a)$};
			\draw[thick, decorate, decoration={brace, amplitude=3pt, raise=3pt, mirror}] (axis cs:\a+\dx, 0) -- (axis cs:{\a+2*\dx}, 0) node[midway, below , yshift=-5pt]{$\Delta x$};
		\end{axis}
	\end{tikzpicture}
	\end{center}
\end{frame}

\begin{frame}[fragile]
	\frametitle{The Definite Integral}
	In the limit where $N\rightarrow\infty$ (equivalently, $\Delta x\rightarrow0$), we get the exact area\footnote{if the function is well behaved...}, and write:
	\begin{equation*}
		\lim\limits_{\Delta x\rightarrow0}\sum\limits_{i=1}^{N}\min\left(f(x_{i}), f(x_{i}+\Delta x)\right)\Delta x = \int\limits_{a}^{b} f(x) \dif x.
	\end{equation*}

	The above sum is called the \emph{lower Darboux sum} of $f$ in the interval $\left[ a,b \right]$.
\end{frame}

% More information here

\begin{frame}
	\frametitle{The Funamental Theorem of Calculus}
	The connection between the two types of integrals mentioned so far is as follows: for a function $f$, its antiderivative $F$ (i.e. $\od{}{x}F=f$) and a real interval $[a,b]$,
	\begin{equation*}
		\int\limits_{a}^{b}f(x)\dif x = F(b) - F(a).
	\end{equation*}

	This is a corollary of the \emph{funamental theorem of calculus}.
\end{frame}

\begin{frame}
	\frametitle{temp}
	\begin{figure}[H]
		\centering
		\begin{tikzpicture}
			\node[operator, draw=col3, fill=col3] (deriv) {$\od{}{x}$};
			\node[left=of deriv] (f1) {$f(x)$};
			\node[right=of deriv] (f1') {$f'(x)$};
			\draw[arrow] (f1) -- (deriv);
			\draw[arrow] (deriv) -- (f1');
			
			\node[operator, draw=col2, fill=col2, below=of deriv] (antideriv) {$\int \dif x$};
			\node[left=of antideriv] (f2) {$f'(x)$};
			\node[right=of antideriv] (f2') {$f(x)+C$};
			\draw[arrow] (f2) -- (antideriv);
			\draw[arrow] (antideriv) -- (f2');
			
			\node[operator, draw=col1, fill=col1, below=of antideriv] (area) {$\int\limits_{a}^{b} \dif x$};
			\node[left=of area] (f3) {$f(x)$};
			\node[right=of area] (S) {$S_{a,b}$};
			\node[above right=of f3, xshift=-5mm, yshift=-5mm] (a) {$a$};
			\node[below right=of f3, xshift=-5mm, yshift=+5mm] (b) {$b$};
			\draw[arrow] (f3) -- (area);
			\draw[arrow] (a) -- (area);
			\draw[arrow] (b) -- (area);
			\draw[arrow] (area) -- (S);
		\end{tikzpicture}
	\end{figure}
\end{frame}
