\section{Linear Transformations and Matrices}

\subsection{Linear Transformations}
Which of the following functions/transformations are linear? Prove your answer by showing that the two criteria for linearity are met (in the case of linear functions), or by giving an example which contraditcs a criterion or a property of linear functions (in the case of non linear functions).
\begin{enumerate}
	\item $f(x) = |x|$.
		\if\withsol1{
			\begin{answer}
				On one hand,
				\begin{equation*}
					f(-4+3) = |-4+3| + |-1| = 1.
				\end{equation*}
				On the other hand,
				\begin{equation*}
					f(-4) + f(3) = |-4| + |3| = 4+3 = 7 \neq 1.
				\end{equation*}
				Thus $f$ is not linear.
			\end{answer}
		}\fi
	
	\item $\sgn(x) = \begin{cases} 1 & \text{ if } x > 0,\\ 0 & \text{ if } x=0,\\ -1 & \text{ if } x<0.\end{cases}$
		\if\withsol1{
			\begin{answer}
				For $x=3,\ \alpha=2$:
				\begin{equation*}
					\sgn(\alpha x) = \sgn(2\cdot 3) = \sgn(6) = 1.
				\end{equation*}
				However,
				\begin{equation*}
					\alpha\cdot\sgn(x) = 2\cdot\sgn(3) = 2\cdot1 = 2 \neq 1.
				\end{equation*}
				Therefore, $\sgn(x)$ is not linear.
			\end{answer}
		}\fi

	\item $T\colvec{2}{x}{y} = \colvec{2}{y}{x}$.
	\if\withsol1{
			\begin{answer}
				\begin{itemize}
					\item \textbf{Scalability}:
						\begin{equation*}
							T\left(\alpha\cdot\colvec{2}{x}{y}\right) = T\left( \colvec{2}{\alpha x}{\alpha y} \right) = \colvec{2}{\alpha y}{\alpha x} = \alpha \colvec{2}{y}{x} = \alpha T\colvec{2}{x}{y}.
						\end{equation*}
					\item \textbf{Additivity}:
						\begin{equation*}
							T\left( \colvec{2}{a}{b} + \colvec{2}{c}{d}\right) = T\colvec{2}{a+c}{b+d} = \colvec{2}{b+d}{a+c} = \colvec{2}{b}{a} + \colvec{2}{d}{c} = T\colvec{2}{a}{b} + T\colvec{2}{c}{d}.
						\end{equation*}
						Therefore, $T$ is linear.
				\end{itemize}
			\end{answer}
		}\fi
	
	\item $T\colvec{2}{x}{y} = \colvec{2}{3x}{2y}+\colvec{2}{1}{-5}$.
		\if\withsol1{
				\begin{answer}
					$T$ does not perserve $\vec{0}$:
					\begin{equation*}
						T\colvec{2}{0}{0} = \colvec{2}{1}{-5},
					\end{equation*}
					and thus is not linear.
				\end{answer}
			}\fi
	
		\item $T\colvec{3}{x}{y}{z} = 3x+y^{2}-z$.
			\if\withsol1{
					\begin{answer}
						\begin{align*}
							T\left( \colvec{3}{x_{1}}{y_{1}}{z_{1}} + \colvec{3}{x_{2}}{y_{2}}{z_{2}} \right) &= T\colvec{3}{x_{1}+x_{2}}{y_{1}+y_{2}}{z_{1}+z_{2}} = 3\left( x_{1}+x_{2} \right) + \left( y_{1}+y_{2} \right)^{2} - \left( z_{1}+z_{2} \right)\\
							&= 3\left( x_{1}+x_{2} \right) + y_{1}^{2} + 2y_{1}y_{2} + y_{2}^{2} + \left( z_{1}+z_{2} \right).
						\end{align*}
						On the other hand,
						\begin{align*}
							T\colvec{3}{x_{1}}{y_{2}}{z_{1}} + T\colvec{3}{x_{2}}{y_{2}}{z_{2}} &= 3x_{1}+y_{1}^{2}-z_{1} + 3x_{2}+y_{2}^{2}-z_{2}\\
							&= 3\left( x_{1}+x_{2} \right) + y_{1}^{2} + y_{2}^{2} + \left( z_{1}+z_{2} \right).
						\end{align*}
						We can see that the difference between the expressions is
						\begin{equation*}
							y_{1}y_{2},
						\end{equation*}
						which means that for any vector for which $y_{1}\neq0$ and $y_{2}\neq0$, the product $y_{1}y_{2}$ is also $\neq0$, and thus the two expressions are not equal. Therefore $T$ is not linear.
					\end{answer}
				}\fi
\end{enumerate}

\if\withsol1{
	\newpage
}\fi

			\subsection{Matrices}
			\begin{enumerate}
				\item The following matrices are defined:
					\begin{equation*}
						A =
						\begin{pmatrix}
							1 & 3  & 7\\
							0 & -4 & 1\\
							2 & 2  & 5\\
						\end{pmatrix},
						\qquad
						B =
						\begin{pmatrix}
							1 & 0 & 1\\
							2 & 0 & 2\\
							3 & 7 & 0\\
						\end{pmatrix},
						\qquad
						C =
						\begin{pmatrix}
							1 & 0 & 0\\
							0 & 1 & 0\\
							0 & 0 & 1\\
						\end{pmatrix}.
					\end{equation*}
					\begin{enumerate}
						\item Calculate the following: $A+B,\ A-C,\ A\cdot B,\ B\cdot A, A\cdot C, C\cdot A$.
							\if\withsol1{
									\begin{answer}
										Adding and subtracting matrices is straight-forward and is simply done element-wise.\\
										Thus:
										\begin{align*}
											A+B &=
											\begin{pmatrix}
												1+1 & 3+0  & 7+1\\
												0+2 & -4+0 & 1+2\\
												2+3 & 2+7  & 5+0\\
											\end{pmatrix}
											=
											\begin{pmatrix}
												2 & 3  & 8\\
												2 & -4 & 3\\
												5 & 9  & 5\\
											\end{pmatrix}\\
											A-C &=
											\begin{pmatrix}
												1-1 & 3-0  & 7-0\\
												0-0 & -4-1 & 1-0\\
												2-0 & 2-0  & 5-1\\
											\end{pmatrix}
											=
											\begin{pmatrix}
												0 & 3  & 7\\
												0 & -5 & 1\\
												2 & 2  & 4\\
											\end{pmatrix}
										\end{align*}
										Multiplying matrices is \underline{\textbf{NOT}} done element-wise, but by calculating each element $c_{\colorbox{col1!25}{i}\colorbox{col2!25}{j}}$ of the resulting matrix as a dot product of the $\colorbox{col1!25}{i}$-th row of $A$ with the $\colorbox{col2!25}{j}$-th column of $B$:\\
						  \begin{pycode}
from scripts import *
A=np.array([[1, 3, 7],[0, -4, 1],[2, 2, 5]])
B=np.array([[1, 0, 1],[2, 0, 2], [3, 7, 0]])
print(r'\begin{tabular}{l c c c c r}')
print(r'& $A$ & $B$ &&&\\')
for i in range(3):
		for j in range(3):
				colordot(A, B, a=i, b=j, col1='col1', col2='col2')
		print(r'\\')
print(r'\end{tabular}')
						  \end{pycode}
										Therefore:
										\begin{equation*}
											A\cdot B = \pyc{draw_matrix(np.dot(A, B))}.
										\end{equation*}
										On the other hand,
										\begin{equation*}
											B\cdot A = \pyc{draw_matrix(np.dot(B, A))}.
										\end{equation*}
										Notice how $A\cdot B \neq B\cdot A$! This is an important property of matrix product: it is generally \textbf{NOT} commutative.

										Continuing:
										\begin{equation*}
											A\cdot C = C\cdot A = \pyc{draw_matrix(A)} = A.
										\end{equation*}

										This is because $C$ is the unitary (from "unit") matrix of size $3$, also known as the \textit{identity matrix}, written as $I$.

										This matrix has the property that for any matrix $A$, the following is always true:
										\begin{equation*}
											I\cdot A = A\cdot I = A.
										\end{equation*}

										The general identity matrix of size-$N$, $I_{N}$ looks as following:
										\begin{align*}
											I_{N}=
											\begin{pmatrix}
												1&0&0&\cdots &0\\0&1&0&\cdots &0\\0&0&1&\cdots &0\\\vdots &\vdots &\vdots &\ddots &\vdots\\0&0&0&\cdots &1
											\end{pmatrix},
										\end{align*}
										i.e. all the main diagonal elements are $1$ while the others are all $0$.
								\end{answer}}
							\fi

						\item For each matrix write its transpose.
							\if\withsol1{
									\begin{answer}
										Transposing a matrix is simply exchanging its rows with columns, i.e. each element $a_{ij}$ becomes $a_{ji}$.\\

										Therefore:
										\begin{pycode}
print(r'\begin{align*}')
print(r'A^{\mathbf{T}}&=')
draw_matrix(np.transpose(A))
print(r',\\B^{\mathbf{T}}&=')
draw_matrix(np.transpose(B))
print(r',\\C^{\mathbf{T}}&=')
draw_matrix(np.identity(3).astype(int))
print(r'.\end{align*}')
										\end{pycode}
								\end{answer}}
							\fi
					\end{enumerate}
				\item Which of the following products $A\cdot B=C$ are defined? Calculate the product if it is possible.

					\begin{enumerate}
						\item $
							\begin{pmatrix}
								1  & 2\\
								-5 & 8\\
							\end{pmatrix}
							\cdot
							\begin{pmatrix}
								2 & 0\\
								0 & 1\\
							\end{pmatrix}
							$
							\if\withsol1{

									\begin{answer}
										\begin{pycode}
A=np.array([[1,2],[-5,8]])
B=np.array([[2,0],[0,1]])
mdot(A, B)
										\end{pycode}
								\end{answer}}
							\fi

						\item $
							\begin{pmatrix}
								1  & 2 & 6\\
								-5 & 8 & 6\\
							\end{pmatrix}
							\cdot
							\begin{pmatrix}
								0 & 1 & -2\\
								1 & 0 & 1\\
								3 & 7 & 2\\
							\end{pmatrix}
							$
							\if\withsol1{

									\begin{answer}
										\begin{pycode}
A=np.array([[1,2,6],[-5,8,6]])
B=np.array([[0,1,-2],[1,0,1],[3,7,2]])
mdot(A, B)
										\end{pycode}
								\end{answer}}
							\fi

						\item $
							\begin{pmatrix}
								2 & 2\\
								7 & 1\\
								0 & 5\\
							\end{pmatrix}
							\cdot
							\begin{pmatrix}
								2 & 0\\
								1 & -1\\
								8 & 1\\
							\end{pmatrix}
							$
							\if\withsol1{

									\begin{answer}
										This multiplication is not possible since the rows of $A$ have $2$ elements each, while the columns of $B$ have $3$ elements each. This is undefined.
								\end{answer}}
							\fi
						\item $
							\begin{pmatrix}
								2 & 2\\
								7 & 1\\
								0 & 5\\
							\end{pmatrix}
							\cdot
							\begin{pmatrix}
								2 & 0\\
								1 & -1\\
								8 & 1\\
							\end{pmatrix}^{\top}
							$
							\if\withsol1{

									\begin{answer}
										This product exists because:
										\begin{equation*}
											\begin{pmatrix}
												2 & 0\\
												1 & -1\\
												8 & 1\\
											\end{pmatrix}^{\mathbf{T}}=
											\begin{pmatrix}
												2 & 1  & 8\\
												0 & -1 & 1\\
											\end{pmatrix},
										\end{equation*}
										and so
										\begin{equation*}
											C = 
											\begin{pmatrix}
												2 & 2\\
												7 & 1\\
												0 & 5\\
											\end{pmatrix}
											\cdot
											\begin{pmatrix}
												2 & 1  & 8\\
												0 & -1 & 1\\
											\end{pmatrix},
										\end{equation*}

										yielding:

										\begin{pycode}
A = np.array([[2,2],[7,1],[0,5]])
B = np.array([[2,0],[1,-1],[8,1]])
mdot(A, np.transpose(B))
										\end{pycode}
								\end{answer}}
							\fi

						\item $
							\begin{pmatrix}
								7 & 0 & 4\\
							\end{pmatrix}
							\cdot
							\begin{pmatrix}
								-3\\
								1 \\
								8 \\
							\end{pmatrix}
							$
							\if\withsol1{
									\begin{answer}
										\begin{pycode}
a = np.array([7, 0, 4])
b = np.array([-3, 1, 8])
vdot(0, 0, a, b, alone=True)
										\end{pycode}
								\end{answer}}
							\fi

						\item $
							\begin{pmatrix}
								1 & 0 & 0\\
								0 & 3 & 0\\
								0 & 0 & -1\\
							\end{pmatrix}
							\cdot
							\begin{pmatrix}
								1\\
								5\\
								5\\
							\end{pmatrix}
							$
							\if\withsol1{
									\begin{answer}
										\begin{pycode}
a = np.array([[1, 0, 0],[0, 3, 0],[0, 0, -1]])
b = np.array([[1],[5],[5]])
mdot(a, b)
										\end{pycode}
								\end{answer}}
							\fi

					\end{enumerate}
			\end{enumerate}
			\if\withsol1{
				\newpage
			\fi

			\subsection{Matrix-Vector Multiplication}
			\begin{enumerate}
				\item The matrix $R_{\theta}=\begin{pmatrix} \cos\left( \theta \right) & -\sin\left( \theta \right) \\ \sin\left( \theta \right) & \cos\left( \theta \right)\\ \end{pmatrix}$ is a 2D-rotation matrix: it rotates any 2D-vector $\vec{v}=\colvec{2}{x}{y}$ by the angle $\theta$ counter clockwise.
					\begin{enumerate}
						\item What do you expect the determinant of $R_{\theta}$ to be? Check your answer via direct calculation.
							\if\withsol1{
									\begin{answer}
										We expect the determinant to be $1$, since rotation does not alter area.\\
										Let's check this via direct calculation:
										\begin{align*}
											\det\left( R_{\theta} \right) &= \begin{pmatrix} \cos\left( \theta \right) & -\sin\left( \theta \right) \\ \sin\left( \theta \right) & \cos\left( \theta \right)\\ \end{pmatrix} \\ 
											&= \cos\left( \theta \right) \cdot \cos\left( \theta \right) - \left(-\sin\left( \theta \right) \cdot \sin\left( \theta \right)\right)\\
											&= \left[\cos\left( \theta \right)\right]^{2} + \left[\sin\left( \theta \right)\right]^{2} \\
											&= 1.
										\end{align*}
									\end{answer}
								}\fi

							\item Show that for $\theta=\ang{180}$ applying the matrix on a vector $\vec{v}=\colvec{2}{x}{y}$ inverts the vector.
								\if\withsol1{
										\begin{answer}
											$R_{\ang{180}}=\begin{pmatrix}
												\cos\left( \ang{180} \right) & -\sin\left( \ang{180} \right) \\
												\sin\left( \ang{180} \right) &  \cos\left( \ang{180} \right) \\
											\end{pmatrix}=
											\begin{pmatrix}
												-1 &	0 \\
												0 & -1 \\
											\end{pmatrix}$

											Applying $R_{\ang{180}}$ on a vector $\vec{v}=\colvec{2}{x}{y}$ results in
											\begin{align*}
												\begin{pmatrix}
													-1 &	0 \\
													0 & -1 \\
												\end{pmatrix}\colvec{2}{x}{y}=
												\begin{pmatrix}
													-x+\cancel{0\cdot y} \\
													\cancel{0\cdot x} -y	\\
												\end{pmatrix}=\colvec{2}{-x}{-y}=-\colvec{2}{x}{y},
											\end{align*}
											which is indeed what we expect from a $\ang{180}$ rotation.
										\end{answer}
									}\fi

									\if\withsol1{
										}\fi

									\item Show that for a vector $\vec{v}=\colvec{2}{x}{y}$ the resulting rotation by $\pm\ang{90}$ using $R$ is orthogonal to $\vec{v}$.
										\if\withsol1{

												\begin{answer}
													Substituting $\ang{90}$ into $R$ yields:
													\begin{align*}  
														R_{\ang{90}}&=
														\begin{pmatrix}
															\cos\left( \ang{90} \right) & -\sin\left( \ang{90} \right) \\
															\sin\left( \ang{90} \right) &  \cos\left( \ang{90} \right) \\
														\end{pmatrix}
														= \begin{pmatrix}
															0 & -1 \\
															1 & 0  \\
														\end{pmatrix}.
													\end{align*}

													Applying $R_{\ang{90}}$ on $\vec{v}=\colvec{2}{x}{y}$ thus yields:
													\begin{align*}
														\begin{pmatrix}
															0 & -1 \\
															1 & 0  \\
														\end{pmatrix}\colvec{2}{x}{y}=
														\begin{pmatrix}
															\cancel{0\cdot x}-1\cdot y \\
															1\cdot x + \cancel{0\cdot y} \\
														\end{pmatrix}=\colvec{2}{x}{y}.
													\end{align*}

													Now let's check that $\colvec{2}{-y}{x}$ is indeed orthogonal to $\colvec{2}{x}{y}$:
													\begin{equation*}
														\colvec{2}{x}{y} \cdot \colvec{2}{-y}{x} = -xy+xy = 0,
													\end{equation*}
													which shows that these two vectors are indeed orthogonal.

													The same for $\theta=-\ang{90}$:
													\begin{align*}  
														R_{-\ang{90}}&=
														\begin{pmatrix}
															\cos\left( -\ang{90} \right) & -\sin\left( -\ang{90} \right) \\
															\sin\left( -\ang{90} \right) &  \cos\left( -\ang{90} \right) \\
														\end{pmatrix}
														= \begin{pmatrix}
															0 & -1 \\
															1 & 0  \\
														\end{pmatrix}.
													\end{align*}

													Applying $R_{-\ang{90}}$ on $\vec{v}=\colvec{2}{x}{y}$ thus yields:
													\begin{align*}
														\begin{pmatrix}
															0 & -1 \\
															1 & 0  \\
														\end{pmatrix}\colvec{2}{x}{y}=
														\begin{pmatrix}
															\cancel{0\cdot x}-1\cdot y \\
															1\cdot x + \cancel{0\cdot y} \\
														\end{pmatrix}=\colvec{2}{x}{y}.
													\end{align*}

													Now let's check that $\colvec{2}{-y}{x}$ is indeed orthogonal to $\colvec{2}{x}{y}$:
													\begin{equation*}
														\colvec{2}{x}{y} \cdot \colvec{2}{-y}{x} = -xy+xy = 0,
													\end{equation*}
													which shows that these two vectors are indeed orthogonal.
												\end{answer}
											}\fi

										\item Show that applying $R_{\theta}$ to a 2D-vector $\vec{v}$ indeed results in a rotation of $\vec{v}$ by $\theta$.
											\if\withsol1{
													\begin{answer}
														Applying the matrix $R(\theta)$ to the vector $\vec{v}=\colvec{2}{v_{x}}{v_{y}}$ results in
														\begin{align*}
															\vec{u} &= R(\theta)\cdot\vec{v}\\
															&= \begin{pmatrix}
																\cos(\theta) & -\sin(\theta)\\
																\sin(\theta) & \cos(\theta)
															\end{pmatrix}\colvec{2}{v_{x}}{v_{y}}\\
															&= \colvec{2}{\cos(\theta)v_{x}-\sin(\theta)v_{u}}{\sin(\theta)v_{x}+\cos(\theta)v_{y}}.
														\end{align*}

														The dot product between $\vec{v}$ and $\vec{u}=R(\theta)\cdot\vec{v}$ is
														\begin{align*}
															\vec{v}\cdot\vec{u} &= v_{x}\left[\cos(\theta)v_{x}-\sin(\theta)v_{y}\right] + v_{y}\left[ \sin(\theta)v_{x}+\cos(\theta)v_{y} \right]\\
															&= v_{x}^{2}\cos(\theta) + v_{y}^{2}\cos(\theta) + v_{x}v_{y}\left[ \cancelto{=0}{\sin(\theta)-\sin(\theta)} \right]\\
															&= \cos(\theta)\left[ v_{x}^{2}+v_{y}^{2} \right]\\
															&= \cos(\theta)\cdot\norm{\vec{v}}^{2}.
														\end{align*}

														This is exactly the definition of a dot product between two vectors of the same norm (in this case $\norm{\vec{v}}$) and an angle $\theta$ between them. This means that the angle between $\vec{v}$ and $\vec{u}=R(\theta)\cdot\vec{v}$ is indeed $\theta$.
													\end{answer}
												}\fi

										\end{enumerate} 
									\item What operation does the following matrix perform? $A = \begin{pmatrix} N & 0 \\ 0 & N \end{pmatrix}$
										\if\withsol1{
												\begin{answer}
													Applying $A$ to a generic 2D-vector $\vec{v}=\colvec{2}{x}{y}$:
													\begin{align*}
														A &= \begin{pmatrix} N & 0 \\ 0 & N \end{pmatrix}\begin{pmatrix} x \\ y \end{pmatrix} \\
														&= \colvec{2}{Nx + \cancel{0\cdot y}}{\cancel{0\cdot x} + Ny}\\
														&= \colvec{2}{Nx}{Ny}\\
														&= N\colvec{2}{x}{y}.
													\end{align*}
													This is simply a scalar multiplication of $\vec{v}$ by $N$ - i.e. scaling $\vec{v}$ by $N$.
												\end{answer}
											}\fi

										\item Construct a $2\times2$ matrix that scales a vector by 2 and then rotates it by $\ang{30}$.
											\if\withsol1{

													\begin{answer}
														Every linear operation on a vector is simply the left-product of a matrix with that matrix (for column vectors. For row vectors this changes to the right-product).\\
														Since scaling and rotating are both linear operations, we can represent them by matrices (as seen in this tutorial) and apply them one after the other.
														Let's first find the value of $R_{\ang{30}}$:
														\begin{align*}
															R_{\ang{30}} &= \rotMat{\ang{30}}\\
															&= \begin{pmatrix} \frac{\sqrt{3}}{2} & -\frac{1}{2} \\ \frac{1}{2} & \frac{\sqrt{3}}{2} \\ \end{pmatrix}\\
														\end{align*}
														Now we can apply both matrices:
														\begin{align*}
															\begin{pmatrix} \frac{\sqrt{3}}{2} & -\frac{1}{2} \\ \frac{1}{2} & \frac{\sqrt{3}}{2} \end{pmatrix} \begin{pmatrix} 2 & 0 \\ 0 & 2 \end{pmatrix} \begin{pmatrix} x \\ y \end{pmatrix}
															&=
															\begin{pmatrix} \frac{\sqrt{3}}{\cancel{2}}\cdot\cancel{2}-\cancel{\frac{1}{2}\cdot0} & \cancel{\frac{\sqrt{3}}{2}\cdot0} - \frac{1}{\cancel{2}}\cdot\cancel{2} \\
																\frac{1}{\cancel{2}}\cdot\cancel{2} - \cancel{\frac{\sqrt{3}}{2}\cdot0} & \cancel{\frac{1}{2}\cdot0} + \frac{\sqrt{3}}{\cancel{2}}\cdot\cancel{2}\\
															\end{pmatrix}
															\begin{pmatrix}
																x \\
																y \\
															\end{pmatrix}\\
															&=
															\begin{pmatrix}  \sqrt{3} & -1 \\
																1 &	\sqrt{3} \\
															\end{pmatrix}
															\begin{pmatrix}
																x \\
																y \\
															\end{pmatrix}
														\end{align*}
														Meaning that
														$\begin{pmatrix}  \sqrt{3} & -1 \\
															1 &	\sqrt{3} \\
														\end{pmatrix}$
														will perform the entire operation of scaling by $2$ and then rotating by $\ang{30}$.\\
														~\\~\\
														\textbf{Moral of the question}: multiplying matrices is equivalent to compositing the respective transformations they represent.
													\end{answer}
												}\fi

											\end{enumerate}
