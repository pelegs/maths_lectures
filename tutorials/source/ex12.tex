\rhead{Exam Preparation}
\section{Exam Preparation}
\subsection{Set Theory}
The following two sets are given:
\begin{equation*}
	A = \left\{ x,y,z \right\},\ B = \left\{ \alpha, \beta, \gamma \right\}.
\end{equation*}
\begin{enumerate}
	\item Write all the elements of the Cartesian Product $A\times B$.
	\if\withsol1{
		\begin{answer}
			The elements of $A\times B$ are pairs of elements $\left( a,b \right)$ such that $a\in A$ and $b\in B$. All these elements are:
			\begin{align*}
				&\left( x,\alpha \right),\ \left( x, \beta \right),\ \left( x,\gamma \right),\\
				&\left( y,\alpha \right),\ \left( y, \beta \right),\ \left( y,\gamma \right),\\
				&\left( z,\alpha \right),\ \left( z, \beta \right),\ \left( z,\gamma \right).\\
			\end{align*}
		\end{answer}
	}\fi

\item Draw two figures: one representing an \textbf{injective} function $\func{f}{A}{B}$, and one representing a \textbf{non-injective} function $\func{g}{A}{B}$. Can $g$ be surjective?

	\if\withsol1{
		\begin{answer}
      \begin{figure}[H]
        \centering
        \begin{tikzpicture}[scale=0.9]
					% Injective function
          \fill[col1!20, draw=col1, thick] (0,0) circle [x radius=0.6, y radius=2]; 
          \node (x) at (0,1.2) {$x$};
          \node (y) at (0,0.0) {$y$};
          \node (z) at (0,-1.2) {$z$};
					\node[above=7.5mm of x] {\color{col1}$A$};

          \fill[col2!20, draw=col2, thick] (2,0) circle [x radius=0.6, y radius=1.75]; 
					\node (alpha) at (2,1.2) {$\alpha$};
          \node (beta)  at (2,0.0) {$\beta$};
          \node (gamma) at (2,-1.2) {$\gamma$};
					\node[above=7.5mm of alpha] {\color{col2}$B$};

          \draw[relation] (x) -- (alpha);
          \draw[relation] (y) -- (beta);
          \draw[relation] (z) -- (gamma);

					\node at (1,3.5) {\underline{$f$, injective}};

				% Non injective function
          \fill[col1!20, draw=col1, thick] (6,0) circle [x radius=0.6, y radius=2]; 
          \node (x2) at (6,1.2) {$x$};
          \node (y2) at (6,0.0) {$y$};
          \node (z2) at (6,-1.2) {$z$};
					\node[above=7.5mm of x2] {\color{col1}$A$};

          \fill[col2!20, draw=col2, thick] (8,0) circle [x radius=0.6, y radius=1.75]; 
					\node (alpha2) at (8,1.2) {$\alpha$};
          \node (beta2)  at (8,0.0) {$\beta$};
          \node (gamma2) at (8,-1.2) {$\gamma$};
					\node[above=7.5mm of alpha2] {\color{col2}$B$};

          \draw[relation] (x2) -- (alpha2);
          \draw[relation] (y2) -- (alpha2);
          \draw[relation] (z2) -- (gamma2);

					\node at (7,3.5) {\underline{$g$, non-injective}};
        \end{tikzpicture}
      \end{figure}
			
			As can be seen, since the function $\func{g}{A}{B}$ is non-injective, there are at least two elements $a_{1},a_{2}\in A$ that are being connected to the same element $b\in B$, which means that at least one element $b_{1}\in B$ is not connected to by an element in $A$, and thus the function can't be surjective.
		\end{answer}
	}\fi
\end{enumerate}

\subsection{Linear Algebra I}
The following three vectors in $\Rs{3}$ are given:
\begin{equation*}
	\vec{a}=\colvec{3}{1}{-2}{2} ,\ \vec{b}=\colvec{3}{6}{4}{1} ,\ \vec{c}=\colvec{3}{-16}{0}{-10}
\end{equation*}
\begin{enumerate}
  \item Calculate the vector $2\vec{a}+\vec{b}$.
  \if\withsol1{
    \begin{answer}
      Multiplying a vector by a scalar is done element-wise. Thus:
      \begin{equation*}
				2\vec{a} = \colvec{3}{1\cdot2}{-2\cdot2}{2\cdot2} = \colvec{3}{2}{-4}{4}.
      \end{equation*}
      Similarly, adding to vectors is done element-wise. Thus:
      \begin{equation*}
				2\vec{a} + \vec{b} = \colvec{3}{2+6}{-4+4}{4+1} = \colvec{3}{8}{0}{5}.
      \end{equation*}
    \end{answer}
  }\fi

  \item Calculate the inner product $\vec{a}\cdot\vec{b}$.
  \if\withsol1{
    \begin{answer}
      The inner product of two vectors is equal to the sum of the element-wise product. Thus:
      \begin{align*}
        \vec{a}\cdot\vec{b} &= 1\cdot6 + (-2)\cdot4 + 2\cdot1\\
        &= 6 - 8 + 2\\
        &= 0
      \end{align*}
    \end{answer}
  }\fi

  \item What is the angle between $\vec{a}$ and $\vec{b}$?
  \if\withsol1{
    \begin{answer}
			The angle $\theta$ between any two vectors can be derived from their inner product: $\cos(\theta) = \frac{\vec{a}\cdot\vec{b}}{\norm{\vec{a}} \norm{\vec{b}}}$.

			In this case, $\vec{a}\cdot\vec{b}=0$. Since $\cos(\ang{90})=0$, the angle between the two vectors is $\ang{90}$ (i.e. they are orthogonal).
    \end{answer}
  }\fi

  \item Are the three vectors $\vec{a},\vec{b},\vec{c}\ $ linearly independent? Prove your answer.
  \if\withsol1{
    \begin{answer}
			Notice that
			\begin{equation*}
				2\vec{a}+\vec{b} = \colvec{3}{8}{0}{5} = -\frac{1}{2}\colvec{3}{-16}{0}{-10} = -\frac{1}{2}\vec{c}.
			\end{equation*}
			Thus the vector $\vec{c}$ can be expressed as a linear combination of $\vec{a}$ and $\vec{b}$ (namely $\vec{c}=-4\vec{a}-2\vec{b}$), and therefore the set is not linearly independent.

			However, sometimes the answer is not that obvious, in which case we can gather the vectors in a matrix, e.g.
			\begin{equation*}
				M =
				\begin{pmatrix}
					1 & 6 & -16\\
					-2 & 4 & 0\\
					2 & 1 & -10
				\end{pmatrix},
			\end{equation*}
			and calculate the determinant of the matrix. If the determinant is $0$, then the vectors are not linearly independent, while if it is not zero, they are linearly independent.

			In this case:
			\begin{align*}
				|M| &= 1\cdot \begin{vmatrix} 4 & 0\\ 1 & -10 \end{vmatrix} -6 \begin{vmatrix} -2 & 0\\ 2 & -10 \end{vmatrix} -16 \begin{vmatrix} -2 & 4\\ 2 & 1 \end{vmatrix}\\
				&= 1\cdot(-40) - 6\cdot20 - 16\cdot(-10)\\
				&= -40 -120 + 160\\
				&= 0,
			\end{align*}
			i.e. the vectors are not linearly independent.
    \end{answer}
  }\fi

  \item Give a geometrical description of the shpae of the set of points in space described by $\left\{ \vec{x}\in\Rs{3}\mid \vec{c}\cdot\vec{x}=0 \right\}$.\\
    (Note: $\vec{c}\cdot\vec{x}$ dontes the inner (scalar) product of $\vec{c}$ and $\vec{x}$)
  \if\withsol1{
    \begin{answer}
      An inner product between two vectors equaling $0$ means that the vectors are orthogonal. Therefore, we are looking for all vectors $\vec{x}$ that are orthogonal to $\vec{c}$. These vectors together form a plane that is itself orthogonal to $\vec{c}$.
    \end{answer}
  }\fi
	
	\item Normalize the vector $\vec{a}$.
	\if\withsol1{
		\begin{answer}
			To normalize a vector, we divide it by its norm, i.e.
			\begin{equation*}
				\hat{v} = \frac{\vec{v}}{\norm{\vec{v}}}.
			\end{equation*}

			In this case:
			\begin{align*}
				\norm{\vec{a}} &= \sqrt{a_{x}^{2} + a_{y}^{2} + a_{z}^{2}}\\
				&= \sqrt{1^{2} + (-2)^{2} + 2^{2}}\\
				&= \sqrt{1 + 4 + 4}\\
				&= \sqrt{9}\\
				&= 3,
			\end{align*}
			and the resulting normalized vector is
			\begin{equation*}
				\hat{a} = \frac{\vec{a}}{\norm{\vec{a}}} = \frac{\vec{a}}{3} = \colvec{3}{\frac{1}{3}}{-\frac{2}{3}}{\frac{2}{3}}.
			\end{equation*}
		\end{answer}
	}\fi
\end{enumerate}

\subsection{Linear Algebra II}
The matrix $M$ represents a transformation that affects $\hat{x}$ and $\hat{y}$ as follows:
\begin{equation*}
	M\hat{x} = \colvec{2}{-1}{8},\quad M\hat{y} = \colvec{2}{8}{11}.
\end{equation*}

\begin{enumerate}
	\item Calculate the eigenvalues and eigenvectors of $M$.
	\if\withsol1{
		\begin{answer}
			First, let us present $M$ explicitely: remember that the $i$-th row of a matrix is the vector to which $\eb{i}$ is mapped by the matrix. Thus, for $M$ in our case, using the maps of $\hat{x}$ and $\hat{y}$ we simply get
			\begin{equation*}
				M =
				\begin{pmatrix}
					-1 & 8 \\
					8 & 11
				\end{pmatrix}.
			\end{equation*}

		Let's now find the eigenvectors and eigenvalues of $M$:
		\begin{align*}
			0 &=
			\begin{vmatrix}
				-1-\lambda & 8 \\
				8 & 11-\lambda
			\end{vmatrix}\\
			&= \left( -1-\lambda \right)\left( 11-\lambda \right) - 64\\
			&= -11+\lambda-11\lambda+\lambda^{2}-64\\
			&= \lambda^{2}-10\lambda-75.
		\end{align*}

		The solution for the above equation is
		\begin{equation*}
			\lambda_{1,2} = \frac{10\pm\sqrt{100+4\cdot75}}{2} = -5,15.
		\end{equation*}

		We will start by finding the eigenvector corresponding to $\lambda_{1}=-5$:
			\begin{equation*}
				\begin{pmatrix}
					-1 & 8\\
					8 & 11
				\end{pmatrix}
				\colvec{2}{x}{y} = \colvec{2}{-5x}{-5y} \Rightarrow
				\left\{
					\begin{array}{ccccc}
						-x & + & 8y  &= & -5x\\
						8x & + & 11y &= & -5y
					\end{array}
				\right.,
		\end{equation*}
				for which the solution is $\frac{x}{y} = -2$, and a representative vector is e.g.
				\begin{equation*}
					\vec{u}_{1} = \colvec{2}{2}{-1}.
				\end{equation*}

				For $\lambda_{2}=15$ we get
				\begin{equation*}
					\begin{pmatrix}
						-1 & 8\\
						8 & 11
					\end{pmatrix}
					\colvec{2}{x}{y} = \colvec{2}{15x}{15y} \Rightarrow
					\left\{
						\begin{array}{ccccc}
							-x & + & 8y  &= & 15x\\
							8x & + & 11y &= & 15y
						\end{array}
					\right.,
				\end{equation*}

				for which the solution is $\frac{x}{y} = \frac{1}{2}$, and a representative vector is e.g.
				\begin{equation*}
					\vec{u}_{2} = \colvec{2}{1}{2}.
				\end{equation*}
	\end{answer}
	}\fi

	\item Does the transformation represented by $M$ flips the orientation of $\Rs{2}$?.
	\if\withsol1{
		\begin{answer}
			The determinant of $M$ is $|M|=\left( -1 \right)\cdot11-8\cdot8 = -11-64 = -75$. This means that the transformation scales space by 75 \textbf{and} flips its orientation.
		\end{answer}
	}\fi
\end{enumerate}


\subsection{Computer Science (Programming)}
The following Java method \LS{f} gets an integer array \LS{x} and a single integer \LS{a} as its arguments:
\inputminted{java}{java/exam.java}
What does the function return?
  \if\withsol1{
    \begin{answer}
      The first two lines of the function (except its header) do the following: set new integer variable \LS{i} to \LS{0}, and setting a boolean variable \LS{b} to \LS{true}:
			\inputminted[firstline=3, lastline=4]{java}{java/exam.java}

      Then, a new loop is run, so long as two conditions are met:
      \begin{enumerate}
        \item \LS{b} is \LS{true} (simply written as \LS{while b}), and
        \item \LS{i} is smaller than the length of the array \LS{x}.
      \end{enumerate}
      If any of these two conditions is not met, the loop stops.\\
      While the loop runs, the function checks whether the \LS{i}-th element in the array (counting the first element as \LS{0}) equals the integer \LS{a} which is an argument passed to the function by the user (see the function's header). This can result in two ways:
      \begin{enumerate}
        \item The \LS{i}-th element indeed equals \LS{a}, in which case \LS{b} is set to \LS{false} (which will cause the loop to finish), or
        \item The \LS{i}-th element does not equal \LS{a}, which will result in simply increasing the value of \LS{i} by one.\\
      \end{enumerate}
          When the while loop is finished, for whatever reason (either for some \LS{i}, the \LS{i}-th element of the array \LS{x} equals to \LS{a}, or the value of \LS{i} reached the size of the array \LS{x}), the function checks the value of \LS{b}. Once again, there are two options:
          \begin{enumerate}
            \item If \LS{b} is true (which means that no element of the array \LS{x} is equal to \LS{a}), there's a problem: the function went over all the elements of the array \LS{x}, and found no element that is equal to \LS{a}. In this case the function prints \LS{Error!} to screen, and returns \LS{-1}.
            \item If \LS{b} is \LS{true} it means that the function found an element of the array \LS{x} that is equal to \LS{a}. It then returns the value of \LS{i}. Since \LS{i} counts the elements of the array \LS{x} until it reaches an element that is equal to \LS{a} (or iterates over the entire array without finding anything), the value of \LS{i} is the position in the array \LS{x} of the element that is equal to \LS{a}.
          \end{enumerate}

          \textbf{To summarize}: the function checks whether there is an element in the array \LS{x} which equals to the integer \LS{a} (given to it as an argument). If there is such element (or more than one), it returns the index of its first occurrence in the array \LS{x}. If there isn't, it returns \LS{-1} and writes \LS{Error!} to the screen.
    \end{answer}
  }\fi


\subsection{Computer Science (Representation of Numbers)}
\begin{enumerate}
  \item What is the binary representation of the decimal number 63?
  \if\withsol1{
    \begin{answer}
      We repeatedly divide $63$ by $2$ until reaching $1$, noting the remainder. Then, we take the resulting remainders from bottom up, and that is the answer:\\~\\
        \begin{pycode}
from scripts import convert, vspace
convert(63,2)
            \end{pycode}
            ~\\~\\
            There is however a quick answer: since $64$ is a power of $2$ ($64=2^{6}$), the binary representation of $64$ is $1$, followed by some number of 0's (in this case 6 zeros). Thus, $63$ would be simply represented by six 1's (think for example of any power of $10$, say $10000$, minus $1$: it's the same behavior, except in the binary case the highest digit is 1 instead of 9).
    \end{answer}
  }\fi

  \item What is the decimal representation of the hexadecimal number 2A5?
  \if\withsol1{
    \begin{answer}
      In hexadecimal numbers, each digit stands for a multiplication of $16$ to the power of $0,1,2,\dots$, counting from the right:\\

      \centering
      \begin{tikzpicture}
        \digits[thick](0.75:16:2A5:3)
        \node[] at (3.5,0.35) {};
      \end{tikzpicture}~\\
      \flushleft

      Thus, the number 2A5 in hexadecimal form is equal to:
      \begin{align*}
        2\times16^{2} + A\times16^{1} + 5\times16^{2} &= 2\times256 + 10\times16 + 5\times1\\
        &= 512 + 160 + 5\\
        &= 677
      \end{align*}
      (Remember that in base-16, $A=10$)
    \end{answer}
  }\fi

  \item Give the 8-bit two's complement representation of the decimal number -84.
  \if\withsol1{
    \begin{answer}
      \begin{pycode}
from scripts import twos
twos(84, 8)
      \end{pycode} 
    \end{answer}
  }\fi
  \item What is the exact value of the fraction which is represented in the ternary base (base-3) by $0.222222222\dots$ (i.e. with infinitely many $2s$ after the period)?
  \if\withsol1{
    \begin{answer}
    Setting $x=0.2222\dots$, we get $3x=2.2222\dots$ (since in base-3 multiplying by 3 is like moving the decimal point one place to the left, same as multiplying by 10 in base-10). But $2.2222\dots$ is like adding $2$ to $0.2222\dots$, and thus we get $3x=2+x$, or simply $x=1$.~\\
    Notice how this is equivalent to the number $0.9999\dots$ in base-10.
    \end{answer}
  }\fi

\end{enumerate}


\subsection{Computer Science (Rule-Based Simulation)}
\begin{enumerate}
  \item Given the following L-system: A $\rightarrow$ (RF)FA,\\
        which string is produced after 3 steps of application of this rule to the start word 'A'?\\
        Draw the graphical structure in the place which is obtained from this string by the following turtle interpretation:
      \begin{itemize}
        \item R: Rotate 45 degrees clockwise.
        \item F: Move forward 1 unit.
        \item A: Do nothing.
        \item (: Add position and angle to stack.
        \item ): Pop position and angle from stack.
      \end{itemize}
  \if\withsol1{
    \begin{answer}
      The rule simply state that that in each step the letter A is replaced with the string '(RF)FA'.\\
      Since in the original string there is just one character, and that is 'A' - after the first step we get (RF)FA.\\
      The second step replaces the single A at the end of the string '(RF)FA' with '(RF)FA', and thus after the second step we get '(RF)F(RF)FA'.\\
      Essentially, each step simply concatenates '(RF)FA' to the end of the string (excluding the 'A'). After the third step this will yield '(RF)F(RF)F(RF)FA'.\\

      The string will evolve as following:
        \begin{pycode}
from scripts import Lsys, turtle_draw2
rules = {'A': '(RF)FA',
         'R': 'R',
         'F': 'F',
         '(': '(',
         ')': ')'}
Ls = [Lsys('A', rules, i) for i in range(4)]
print('\\begin{enumerate}[label={\\arabic*.},start=0]')
for L in Ls:
    print('\\item', L)
print('\\end{enumerate}')
        \end{pycode}

      Using the given drawing rules, the total string '(RF)F(RF)F(RF)FA' yields the following drawing:~\\~\\

      \centering
      \begin{tikzpicture}[scale=0.75]
        \begin{pycode}
from scripts import Lsys, turtle_draw2
rules = {'A': '(RF)FA',
         'R': 'R',
         'F': 'F',
         '(': '(',
         ')': ')'}
Ls = Lsys('A', rules, 3)
turtle_draw2(lst=Ls)
        \end{pycode}
      \end{tikzpicture}
    \end{answer}
  }\fi

  \item A modified L-system is given: A $\rightarrow$ [RFA]FA.\\
        Draw the corresponding graphical structure after 3 steps.
  \if\withsol1{
    \begin{answer}
      Now there are two 'A's to replace at each step, and each replacement generates two new 'A's. Thus, the system will grow as follows:
        \begin{pycode}
from scripts import Lsys, turtle_draw2
rules = {'A': '(RFA)FA',
         'R': 'R',
         'F': 'F',
         '(': '(',
         ')': ')'}
Ls = [Lsys('A', rules, i) for i in range(4)]
print('\\begin{enumerate}[label={\\arabic*.},start=0]')
for L in Ls:
    print('\\item', L)
print('\\end{enumerate}')
        \end{pycode}

      And the drawing after three steps would be:~\\~\\

      \centering
      \begin{tikzpicture}[scale=0.75]
        \begin{pycode}
from scripts import Lsys, turtle_draw2
rules = {'A': '(RFA)FA',
         'R': 'R',
         'F': 'F',
         '(': '(',
         ')': ')'}
Ls = Lsys('A', rules, 3)
turtle_draw2(lst=Ls)
        \end{pycode}
      \end{tikzpicture}
    \end{answer}
  }\fi

  \item How does the number of single lines (obtained from an "F" symbol) grow (quantitatively) with the number of steps for L1? For L2?
  \if\withsol1{
    \begin{answer}
      In the case of L1, each step $N$ yields two 'F's. Thus, the change is linear: $\text{Number of lines}=2N$.\\ 
      In the case of L2, we get the following number of 'F's in each successive step: $0, 2, 6, 14$. We can notice that these numbers are each a distance of $2$ away from $2^{N+1}$ (i.e. $2, 4, 8, 16$). The growth is exponential, and given by $\text{Number of lines}=2^{N+1}-2$.
    \end{answer}
  }\fi
\end{enumerate}



\subsection{Calculus (Sequences)}
\begin{enumerate}
	\item The following sequence is given:
	\begin{equation*}
		a_{n} = \frac{4-n}{2n+3}.
	\end{equation*}

	Is the sequence increasing, decreasing or neither for $n\geq10$? Is it converging on a finite value, and if so - what is this value?
	\if\withsol1{
		\begin{answer}
			We can check the ratio sequence:
			\begin{align*}
				\frac{a_{n+1}}{a_{n}} &= \frac{\frac{4-(n+1)}{2(n+1)+3}}{\frac{4-n}{2n+3}}\\
				&= \frac{\frac{4-n-1)}{2n+2+3}}{\frac{4-n}{2n+3}}\\
				&= \frac{\frac{3-n}{2n+5}}{\frac{4-n}{2n+3}}\\
				&= \frac{\left( 3-n \right)\left( 2n+3 \right)}{\left( 2n+5 \right)\left( 4-n \right)}\\
				&= \frac{6n+9-2n^{2}-3n}{8n-2n^{2}+20-5n}\\
				&= \frac{-2n^{2}+3n+9}{-2n^{2}-5n+20}.
			\end{align*}

			For $n\geq10$, the fraction $\frac{a_{n+1}}{a_{n}}<1$, and thus the sequence is decreasing.

			For $n\rightarrow\infty$, we can ignore $4$ in the numerator and $3$ in the denomenator, and see that the sequence approaches $\frac{-n}{2n} = -\frac{1}{2}$, which is its limit.
		\end{answer}
	}\fi

	\item Suppose that the following sequence converges:
		\begin{equation*} 
			a_{n+1} = a_{n}^{2} + \frac{1}{4},\quad a_{1}=\frac{1}{4}.
		\end{equation*}
			What is its limit as $n\longrightarrow\infty$?
		\if\withsol1{
			\begin{answer}
				Since $a_{n}$ converges, both $a_{n+1}$ and $a_{n}$ will converge to the same limit $L$ as $n\rightarrow\infty$. Thus, we get
				\begin{equation*}
					L= L^{2} + \frac{1}{4},
				\end{equation*}
				which when substituted into the quadratic formula yields
				\begin{equation*}
					L = \frac{1\pm\sqrt{1-4\cdot\frac{1}{4}}}{2} = \frac{1}{2}.
				\end{equation*}
			\end{answer}
		}\fi
\end{enumerate}


\subsection{Calculus (Univariate Functions, Limits)}
Evaluate the following limit:
\begin{equation*}
	\lim\limits_{x\rightarrow-\infty}xe^{x}.
\end{equation*}
\if\withsol1{
	\begin{answer}
		We can see that
		\begin{equation*}
			\lim\limits_{x\rightarrow-\infty}x = -\infty,\quad \lim\limits_{x\rightarrow-\infty}e^{x} = 0,
		\end{equation*}
		i.e. we have a situation of $-\infty\cdot0$. To solve this, we will use L'H\^{o}pital rule. First we will re-write the expression:
		\begin{equation*}
			xe^{x} = \frac{x}{e^{-x}}.
		\end{equation*}

		Now we can derive the top and bottom parts separately:
		\begin{equation*}
			x'=1,\quad \left(e^{-x}\right)'=-e^{-x}.
		\end{equation*}

		This yields
		\begin{align*}
			\lim\limits_{x\rightarrow-\infty}xe^{x} &= \lim\limits_{x\rightarrow-\infty}\frac{x}{e^{-x}}\\
			\tikznode{eq}{} &= \lim\limits_{x\rightarrow-\infty} \frac{1}{-e^{-x}}\\
			&= 0.
		\end{align*}
		
		\begin{tikzpicture}[overlay, remember picture]
			\node[below left=of eq, xshift=-1cm, col4] (eqtxt) {L'H\^{o}pital rule};
			\draw[->, >=stealth, col4] (eqtxt.east) to [out=0, in=180] ($(eq.west) + (0,1mm)$);
		\end{tikzpicture}
	\end{answer}
}\fi

\subsection{Calculus (Univariate Functions, Differentiation)}
Given the following function: $f\left( x \right)=\frac{1}{3}x^{3}-4x^{2}+7x-5$,\\
\begin{enumerate}
  \item Find all $x$ values where the function $f$ has local extrema and classify them as minima or maxima.
  \if\withsol1{
    \begin{answer}
      Local extrema are points where the derivative of a function equals $0$. Thus, we solve for $f'\left( x \right)=0$:
      \begin{align*}
        f'\left( x \right) &= \frac{3}{3}x^{2}-2\cdot4x+7\\
        &= x^{2}-8x+7\\
        &\Downarrow\\
        0 &= x^{2}-8x+7\\
        &\Downarrow\\
        x_{1,2} &= \frac{8\pm\sqrt{64-28}}{2}\\
        &= \frac{8\pm\sqrt{36}}{2}\\
        &= \frac{8\pm6}{2}\\
        &= 4\pm3\\
        &= 1, 7
      \end{align*}
      In order to classify the extrema, we need to use the second derivative of $f\left( x \right)$: $f''\left( x \right) = 2x-8$.
      Substituting point $x=1,\ x=7$ in $f''\left( x \right)$ yields:
      \begin{enumerate}
        \item $f''\left( x=1 \right)=2-8=-6<0$, and thus at $x=1$ there is a local maximum.
        \item $f''\left( x=7 \right)=14-8=6>0$, and thus at $x=7$ there is a local minimum.
      \end{enumerate}
    \end{answer}
  }\fi

  \item Find where the function is increasing/decreasing, and all $x$ values of inflection points.
  \if\withsol1{
    \begin{answer}
      There are two extrema (at $x=1$ and $x=7$), and thus we need to check three domains:
      \begin{enumerate}
        \item $x<1$
        \item $1<x<7$
        \item $x>7$
      \end{enumerate}
      Since at $x=1$ the function is at a maximum, we expect that for $x<1$ the function will increase. Let's verify it by substituting some $x<1$ into $f'\left( x \right)$, say $x=0$: $f'\left( x=0 \right) = \cancel{0^{2}}-\cancel{8\cdot0} + 7 = 7$. Indeed $7$ is positive, and thus for $x<1$ the function increases.\\

      Similarly, we expect the function to decrease for $1<x<7$. Checking by substituting $x=2$ into $f'\left( x \right)$: $f'\left( x=2 \right)=2^{2}-8\cdot2+7=4-16+7=-5$. Indeed, this shows that the function is decreasing for $1<x<7$.\\

      Last but not least, for $x>7$ we expect the function to increase again (owing to the minimum at $x=7$). By substituting $x=10$ we can see that $f'\left( x=10 \right)=10^{2}-8\cdot10+7=100-80+7=27$, i.e. - the function indeed increases for this domain.\\

      Summarizing:
      \begin{enumerate}
        \item $x<1$: $f\left( x \right)$ increases.
        \item $1<x<7$: $f\left( x \right)$ decreases.
        \item $x>7$: $f\left( x \right)$ increases.
      \end{enumerate}

      A plot of the function in the relevant domain, showing the extrema:
      \begin{figure}[H]
        \centering
        \begin{tikzpicture}[]
								\begin{axis}[
										Axis Style,
										xmin=-3, xmax=11,
										ymin=-50, ymax=10,
										domain=-3:11,
										x label style={at={(axis description cs:1.05,0.875)}, anchor=north},
										y label style={at={(axis description cs:0.21,1.025)}, anchor=south},
									]
            \pgfplotsset{grid style={dashed,gray!30}}
            \addplot[function, col1] plot (\x, { 0.334*\x^3-4*\x^2+7*\x-5 } );
          \end{axis}
        \end{tikzpicture} 
      \end{figure}
    \end{answer}
  }\fi
\end{enumerate}

